\documentclass {beamer}

\usepackage{amsmath,amssymb}


%
% Custom operator declarations
%
\DeclareMathOperator{\ZZ}{\mathbb{Z}}
\DeclareMathOperator{\CC}{\mathbb{C}}
\DeclareMathOperator{\hh}{\mathfrak{h}}
\DeclareMathOperator{\re}{\text{Re}}
\DeclareMathOperator{\im}{\text{Im}}
\newcommand{\thetachar}[2] {\theta {\scriptsize \begin{bmatrix}#1\\#2\end{bmatrix}}}
\newcommand{\thetacharsm}[2] {\theta \left[ \begin{smallmatrix} #1
      \\ #2 \end{smallmatrix} \right]}


%
% Custom theme
%
\usetheme{AAUsimple}


%%%%%%%%%%%%%%%%%%%%%%%%%%%%%%%%%%%%%%%%%%%%%%%%%%%%%%%%%%%%%%%%%%%%%%%%%%%%%%%
\title{Algebro-Geometric Approach to Nonlinear Integrable Equations}
\subtitle{General Examination}
\author{
  Chris Swierczewski\\
  {\tt cswiercz@uw.edu}
}
\institute{
  Department of Applied Mathematics\\
  University of Washington\\
  Seattle, Washington
}
\pgfdeclareimage[height=1.5cm]{titlepagelogo}{AAUgraphics/aau_logo_new_circle}
\titlegraphic{
  \pgfuseimage{titlepagelogo}
}

\AtBeginSection[]{\begin{frame}\frametitle{Table of Contents}\tableofcontents[currentsection,subsectionstyle=show/show/shaded]\end{frame}}
%%%%%%%%%%%%%%%%%%%%%%%%%%%%%%%%%%%%%%%%%%%%%%%%%%%%%%%%%%%%%%%%%%%%%%%%%%%%%%%


% Ideas:

% o add a seriers of cartoons demonstrating the concept of
% homotopy. relate to integration.



%%%%%%%%%%%%%%%%%%%%%%%%%%%%%%%%%%%%%%%%%%%%%%%%%%%%%%%%%%%%%%%%%%%%%%%%%%%%%%%
\begin{document}
%%%%%%%%%%%%%%%%%%%%%%%%%%%%%%%%%%%%%%%%%%%%%%%%%%%%%%%%%%%%%%%%%%%%%%%%%%%%%%%

\begin{frame}[plain,noframenumbering]
  \titlepage
\end{frame}

\begin{frame}{Table of Contents}{}
  \tableofcontents
\end{frame}

%%%%%%%%%%%%%%%%%%%%%%%%%%%%%%%%%%%%%%%%%%%%%%%%%%%%%%%%%%%%%%%%%%%%%%%%%%%%%%%
\section{Introduction}
%%%%%%%%%%%%%%%%%%%%%%%%%%%%%%%%%%%%%%%%%%%%%%%%%%%%%%%%%%%%%%%%%%%%%%%%%%%%%%%

%------------------------------------------------------------------------------
\subsection{Integrable Equations and Theta Functions}
%------------------------------------------------------------------------------

\begin{frame}{The Kadomtsev--Petviashvili Equation}{}
  \vspace{10pt}
  $u(x,y,t) = $ surface height of a 2d periodic shallow water wave.
  \[
  \tfrac{3}{4} u_{yy} = \frac{\partial}{\partial x} \left(
  u_t - \tfrac{1}{4} \left(6uu_x + u_{xxx}\right) \right)
  \]

  \begin{figure}
    \centering
    \includegraphics[height=0.4\textheight]{images/livekp.jpg}
    \caption{KP waves off the coast of \^{I}le de R\'{e}, France.}
  \end{figure}
\end{frame}

\begin{frame}{Theta Function Solutions}{}
  KP admits a (large) family of solutions of the form
  \[
  u(x,y,t) = 2 \partial_x^2 \log
  \theta(Ux + Vy + Wt + z_0, \Omega)
  \]
  Where, for any $g \in \ZZ_+$,
  \begin{itemize}
    \item $U,V,W,z_0 \in \CC^g$,
    \item $\Omega \in \CC^{g \times g}$,
    \item $\theta : \CC^g \times \CC^{g \times g} \to \CC$
      is the {\it``Riemann theta function''}.
  \end{itemize}

  \vspace{20pt}

  \begin{block}
  {\it In fact, Theta function solutions are \underline{dense} in the
    space of periodic solutions to KP.}
  \end{block}
\end{frame}

\begin{frame}{Theta Function Solutions}
  The Riemann theta function $\theta:\CC^g \times \CC^{g \times g} \to
  \CC$
  \[
  \theta(z,\Omega) = \sum_{n \in \ZZ^g}
  e^{
    2 \pi i
    \left( \tfrac{1}{2} n \cdot \Omega n + n \cdot z \right)
  }
  \]
  \begin{block}{Convergence}
    $\im(\Omega)$ must be positive definite. Additionally, we only
    consider symmetric $\Omega$. So called space of ``Riemann
    matrices'': $\hh_g$.
  \end{block}
  \[
    \theta:\CC^g \times \hh_g \to \CC
  \]
\end{frame}

\begin{frame}{Demo}{}
  \begin{center}
    Demo: Riemann theta functions.
  \end{center}
\end{frame}



%------------------------------------------------------------------------------
\subsection{Connections to Algebraic Geometry}
%------------------------------------------------------------------------------



\begin{frame}{Connections to Algebraic Geometry}{}
  \[
    \text{KP:} \quad u(x,y,t) = 2 \partial_x^2 \log
    \theta(Ux + Vy + Wt + z_0, \Omega)
  \]
  The constants $U,V,W,z_0,\Omega$ are not arbitrary. They come from a
  ``complex algebraic curve'': given a polynomial $f(X,Y) \in
  \mathbb{C}[X,Y]$
  \[
  F(X,Y) = a_d(X)Y^d + a_{d-1}(X)Y^{d-1} + \cdots + a_0(X)
  \]
  the curve $\Gamma$ is the set
  \[
  \Gamma = \left\{ (X,Y) \in \CC^2 \, : \, F(X,Y) = 0 \right\}.
  \]

  \begin{block}
  {This talk: where does the Riemann matrix $\Omega$ come from and how
    do we compute it?}
  \end{block}
\end{frame}

\begin{frame}{A Simple Example}{}
\end{frame}



%%%%%%%%%%%%%%%%%%%%%%%%%%%%%%%%%%%%%%%%%%%%%%%%%%%%%%%%%%%%%%%%%%%%%%%%%%%%%%%
\section{Computing on Riemann Surfaces}
%%%%%%%%%%%%%%%%%%%%%%%%%%%%%%%%%%%%%%%%%%%%%%%%%%%%%%%%%%%%%%%%%%%%%%%%%%%%%%%

%------------------------------------------------------------------------------
\subsection{Algebra}
%------------------------------------------------------------------------------

\begin{frame}{Holomorphic Differentials}{}
  Holomorphic Differentials
\end{frame}

%------------------------------------------------------------------------------
\subsection{Geometry}
%------------------------------------------------------------------------------

\begin{frame}{Monodromy}{}
  Monodromy
\end{frame}


%------------------------------------------------------------------------------
\subsection{{\tt abelfunctions}}
%------------------------------------------------------------------------------

\begin{frame}[fragile]{{\tt abelfunctions}}{}
  {\tt abelfunctions}: software for computing with Abelian functions.
  \begin{itemize}
    \item Riemann Theta functions,
    \item period matrices,
    \item Abel map.
  \end{itemize}

  \begin{verbatim}
    >>> from abelfunctions import *
    >>> from sympy.abc import x,y
    >>> f = y**3 - 2*x**3*y + x**7
    >>> X = RiemannSurface(f,x,y)
    >>> A,B = X.period_matrix()
  \end{verbatim}

  \begin{centering}
    \small
    Available at: http://github.com/cswiercz/abelfunctions
  \end{centering}
  \begin{center}

  \end{center}
\end{frame}


%%%%%%%%%%%%%%%%%%%%%%%%%%%%%%%%%%%%%%%%%%%%%%%%%%%%%%%%%%%%%%%%%%%%%%%%%%%%%%%
\section{Thesis Goals and Applications}
%%%%%%%%%%%%%%%%%%%%%%%%%%%%%%%%%%%%%%%%%%%%%%%%%%%%%%%%%%%%%%%%%%%%%%%%%%%%%%%

%------------------------------------------------------------------------------
\subsection{Solving Integrable PDEs}
%------------------------------------------------------------------------------



%------------------------------------------------------------------------------
\subsection{Computing with Abelian Functions}
%------------------------------------------------------------------------------

\begin{frame}{Meromorphic Functions for $g=0$}{}
  All meromorphic functions on $\mathbb{C}^*$ are of the form
  \[
  f(x) = \frac{p(x)}{q(x)} =
  \frac{(x-a_1) \cdots (x-a_n)}{(x-b_1) \cdots (x-b_n)}.
  \]
  \begin{itemize}
    \item Zeros are the roots of $p$: $a_1, \ldots, a_n$.
    \item Poles are the roots of $q$: $b_1, \ldots, b_n$.
  \end{itemize}
  (By Residue Theorem, \#\!\! roots $=$ \#\!\! poles.)

  \vspace{16pt}

  Is there an analogue to $(x-a_i)$ and $(x-b_j)$ on higher genus
  surfaces?

  \vspace{16pt}

  \begin{block}
  {\footnotesize Aside: Why is the Weierstrass $\wp$ function not
    meromorphic on $\mathbb{C}^*$?}
  \end{block}
\end{frame}

\begin{frame}{Meromorphic Functions for $g>0$}{}
  $X$ : complex curve of genus $g > 0$
  \[
      f(P) = \prod_{i=1}^n \frac{E(P,A_i)}{E(P,B_i)}
  \]
  where $E(P,Q)$ is ``Fay's Prime Form''
  \[
      E(P,Q) = \frac{
        \thetachar{\alpha}{\beta}
        \left(
        \int_P^Q \boldsymbol{\omega}
        \right)
        }
      {
        \sqrt{\zeta(P)} \sqrt{\zeta(Q)}
      }
  \]
  and $\zeta : X \to H^1(X)$ is the holomorphic differential
  \[
      \zeta = \nabla \thetachar{\alpha}{\beta} \Big|_{z=0} \cdot
      \boldsymbol{\omega}
  \]
\end{frame}

%------------------------------------------------------------------------------
\subsection{The Constructive Schottky Problem}
%------------------------------------------------------------------------------

\begin{frame}{Schottky Problem}{}
  Counting Riemann matrices $\Omega = X + iY$.
  \begin{itemize}
    \item In general: $g(g+1)/2$ free parameters.
    \item Period Matrices: $3g-3$ free parameters.
  \end{itemize}

  \begin{block}
    {Q: \it For $g \geq 3$, when is a Riemann matrix $\Omega$ a period
      matrix?}
  \end{block}
  A: When $u(x,y,t) = 2 \partial^2_x \log \theta(Ux + Vy + Wt +
  z_0,\Omega)$ solves the KP Equation.
\end{frame}


\begin{frame}[plain,noframenumbering]
  \finalpage{
    {\huge Thank you!} \\
    \vspace{20pt}{\tt \scriptsize cswiercz@uw.edu} \\
    {\tt \scriptsize https://github.com/cswiercz/abelfunctions}
    {\tt \scriptsize https://github.com/cswiercz/general-exam}
  }
\end{frame}

\end{document}
