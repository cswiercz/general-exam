\documentclass {beamer}

% Note: this presentation uses a custom theme
\usetheme{AAUsimple}

\title{Algebro-Geometric Approach to Nonlinear Integrable Equations}
\subtitle{General Examination}
\author{
  Chris Swierczewski\\
  {\tt cswiercz@uw.edu}
}
\date{\today}
\institute{
  Department of Applied Mathematics\\
  University of Washington\\
  Seattle, Washington

}
\pgfdeclareimage[height=1.5cm]{titlepagelogo}{AAUgraphics/aau_logo_new_circle}
\titlegraphic{
  \pgfuseimage{titlepagelogo}
}


%%%%%%%%%%%%%%%%%%%%%%%%%%%%%%%%%%%%%%%%%%%%%%%%%%%%%%%%%%%%%%%%%%%%%%%%%%%%%%%
\begin{document}
%%%%%%%%%%%%%%%%%%%%%%%%%%%%%%%%%%%%%%%%%%%%%%%%%%%%%%%%%%%%%%%%%%%%%%%%%%%%%%%

\begin{frame}[plain,noframenumbering]
  \titlepage
\end{frame}

\begin{frame}{Table of Contents}{}
  \tableofcontents
\end{frame}

%%%%%%%%%%%%%%%%%%%%%%%%%%%%%%%%%%%%%%%%%%%%%%%%%%%%%%%%%%%%%%%%%%%%%%%%%%%%%%%
\section{Introduction}
%%%%%%%%%%%%%%%%%%%%%%%%%%%%%%%%%%%%%%%%%%%%%%%%%%%%%%%%%%%%%%%%%%%%%%%%%%%%%%%

%------------------------------------------------------------------------------
\subsection{Integrable Equations}
%------------------------------------------------------------------------------

\begin{frame}{The Kadomtsev--Petviashvili Equation}{}
  $u(x,y,t) = $ surface height of shallow water wave.
  \[
  \tfrac{3}{4} u_{yy} = \frac{\partial}{\partial x} \left(
  u_t - \tfrac{1}{4} \left(6uu_x + u_{xxx}\right) \right)
  \]
  \begin{itemize}
    \item 2D shallow water wave propagation.
  \end{itemize}
\end{frame}

\begin{frame}{The Kadomtsev--Petviashvili Equation}{}
  [image]
\end{frame}


\begin{frame}{Integrable Equations}{}
  KP is {\it integrable}:
\end{frame}

\section{Results}



%%%%%%%%%%%%%%%%%%%%%%%%%%%%%%%%%%%%%%%%%%%%%%%%%%%%%%%%%%%%%%%%%%%%%%%%%%%%%%%
\section{Future Work}
%%%%%%%%%%%%%%%%%%%%%%%%%%%%%%%%%%%%%%%%%%%%%%%%%%%%%%%%%%%%%%%%%%%%%%%%%%%%%%%

\begin{frame}{Schottky Problem}{}
  Counting Riemann matrices $\Omega = X + iY$.
  \begin{itemize}
    \item In general: $g(g+1)/2$ free parameters.
    \item Period Matrices: $3g-3$ free parameters.
  \end{itemize}

  Q: {\it ``For $g \geq 3$, when is a Riemann matrix $\Omega$ a period
    matrix?}

  A: When $u(x,y,t) = 2 \partial^2_x \log \theta(z,\Omega)$ solves the
  KP Equation.
\end{frame}


\begin{frame}[plain,noframenumbering]
  \finalpage{Thank you!\\\vspace{20pt}{\tt \scriptsize cswiercz@uw.edu}}
\end{frame}

\end{document}
