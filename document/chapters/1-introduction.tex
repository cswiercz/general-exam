%%%%%%%%%%%%%%%%%%%%%%%%%%%%%%%%%%%%%%%%%%%%%%%%%%%%%%%%%%%%%%%%%%%%%%%%%%%%%%%
\section{Introduction}
%%%%%%%%%%%%%%%%%%%%%%%%%%%%%%%%%%%%%%%%%%%%%%%%%%%%%%%%%%%%%%%%%%%%%%%%%%%%%%%

The Kadomtsev-Petviashvili (KP) equation is a partial differential
equation used to describe the surface height of a two-dimensional
periodic shallow water wave. Depending on certain physical
considerations [XXX], which we will ignore, one can derive either of the
following two equations
\begin{align}
  \left(-4u_t + 6uu_x + u_{xxx}\right)_x + 3\sigma^2 u_{yy} = 0, \quad
  \sigma^2 = -1, \label{eqn: KP1} \\
  \left(-4u_t + 6uu_x + u_{xxx}\right)_x + 3\sigma^2 u_{yy} = 0, \quad
  \sigma^2 = +1. \label{eqn: KP2}
\end{align}
where $u(x,y,t)$ is the surface height as a function of position $(x,y)$
and time $t$. In the sequel we will not rely on this distinction and
simply refer to the ``KP equation''.

%% \marginpar{
%%   \includegraphics[width=\marginparwidth]{images/livekp.jpg}
%%   \captionof{figure}{KP waves off the coast of \^{I}le de R\'{e}, France.}
%% }

The KP equation admits a large family of quasiperiodic solutions of the
form
\begin{equation} \label{eqn: kpsol}
  u(x,y,t) = 2 \partial_x^2 \log \theta(Ux+Vy+Wt+z_0, \Omega)
\end{equation}
where $\theta$ is the Riemann theta function.

\begin{definition} \label{def: riemanntheta}
  The {\bf Riemann theta function} $\theta: \CC^g \times \hh_g \to \CC$
  is defined in terms of its Fourier series
  \begin{equation} \label{eqn: riemanntheta}
    \theta(z,\Omega) = \sum_{n \in \ZZ^g}
    e^{2 \pi i \left( \tfrac{1}{2} n \cdot \Omega n + n \cdot z \right)}.
  \end{equation}
  $\theta$ converges absolutely and uniformly on compact sets in $\CC^g
  \times \hh_g$ where $\hh_g$ is the space of all {\it ``Riemann
    matrices''} --- complex symmetric matrices with positive definite
  imaginary part.
\end{definition}

From the definition, we see that the Riemann theta function is periodic
in $z$ with integer periods and quasi-periodic in $z$ in the columns of
$\Omega$. In other words, if $m,n \in \ZZ^g$ then
\begin{equation} \label{eq: quasiperiodicity}
    \theta(z + m + \Omega n, \Omega) =
    e^{-2 \pi i \left( \tfrac{1}{2} n \cdot \Omega n + n \cdot z \right) }
    \theta(z, \Omega).
\end{equation}

A generalization of the Riemann theta function, involving a non-integer
shift in some of its arguments, is referred to as the Riemann theta
function with characteristics.

\begin{definition} \label{def: thetachar}
Let $\alpha,\beta \in [0,1)^{g}$. The {\bf Riemann theta function with
characteristic $\thcharsm{\alpha}{\beta}$} is defined as
\begin{align*}
  \theta\thchar{\alpha}{\beta}(z, \Omega) &=
  \sum_{n \in \mathbb{Z}^g}
  e^{2 \pi i \left( \tfrac{1}{2} (n+\alpha) \cdot \Omega (n+\alpha) +
    (n + \alpha) \cdot (z + \beta) \right) } \\
  &=
  e^{2\pi i \left( \tfrac{1}{2} \alpha \cdot \Omega \alpha +
    \alpha \cdot (z + \beta) \right)}
  \theta(z + \Omega \alpha + \beta, \Omega).
\end{align*}
\end{definition}

Note that $\theta \thcharsm{0}{0}(z,\Omega) = \theta(z,\Omega)$. See
\cite{DLMF,MumfordI07,MumfordII07} for further definitions and
properties of the Riemann theta function.

Returning to Equation \eqref{eqn: kpsol}, these solutions are the
so-called ``theta function solutions'' to the KP equation and families
of such solutions exist for every $g > 0$. In fact, the totality of
solutions of this form are dense the space of all periodic solutions to
KP. [XXX Marchenko and Ostrovskii?, see Dubrovin] The constants
$U,V,W,z_0 \in \CC^g$ and $\Omega \in \hh_g$, as well as the ``genus''
$g$, are all determined from a Riemann surface corresponding to a
complex plane algebraic curve $f \in \CC[x,y]$ and any such curve can
produce a solution to KP. \cite{Dubrovin81} We will postpone the
definition of these constants in terms of known quantities until more
machinery is developed in the following chapters.

\marginpar{Improve / elaborate upon the role of Abelian functions. This
  paragraph is a bit...I dunno.}

In general, periodic solutions to partial differential equations are
Abelian functions: doubly-periodic meromorphic functions defined on
Riemann surfaces [XXX]. Riemann theta functions play a central role in
the theory of Abelian functions in that all Abelian functions can be
written as a rational function of the Riemann theta function and its
derivatives. (Such as in the KP solution above.) Therefore, a primary
focus of study is the method of computation of these Abelian functions.

The theory of Riemann theta functions and algebraic curves is applicable
to solving problems in fields other than nonlinear partial differential
equations. For example Riemann theta functions are involved in the
computation of bitangent lines; useful for computations in
optimization-related fields such as algebraic geometry and convex
optimization. In algebraic geometry, bitangents can represent smooth
complex plane quartic curves as both a symmetric determinant of a linear
form (or, determinantal representation) as well as a sum of three
squares. \cite{PSV11} In convex optimization, bitangents are used to
construct a ``visibility complex'' which, in turn, is used to solve the
shortest path problem in Euclidean space \cite{PocchiolaVegter93}

\begin{definition} \label{def: bitangent}
  A {\bf bitangent} to a plane algebraic curve $C : f(x,y) = 0, f \in
  \CC[x,y]$ is a line $\mathcal{L} \subset \CC$ that lies tangent to $C$
  at at least two distinct points.
\end{definition}

By Bezout's Theorem, if a curve has a bitangent it necessarily must be
of degree at least four. \cite{Bezout1779} A result of Pl\"{u}cker
determines that a degree four complex curve admits exactly 28 complex
bitangents. \cite{Plucker34} In particular, Pl\"{u}cker showed that the
number of real bitangents of any quartic must be 28, 16, or fewer than
9. The connection between Riemann theta functions and the bitangent
lines of smooth quartics was known to Riemann \cite{Riemann76} and, in
fact, can be computed using the tools developed in this research. See
Figure \ref{fig: edge} for an example.

\marginpar{Note: Figure \ref{fig: edge} will be replaced with the one
  generated by abelfunctions once I have it working properly again /
  once I find the image on my computer.}

\begin{figure}[t]
  \centering
  \includegraphics[width=0.8\textwidth]{images/edgequartic}
  \caption{The real graph of the Edge Quartic $C: f(x,y) = 25(x^4+y^4+1)
    - 34(x^2y^2+x^2+y^2) = 0$ and its 28 real bitangents. Note that four
    of them lie tangent $C$ at infinity. These lines were computed using
    the Riemann theta function.}
  \label{fig: edge}
\end{figure}

Finally, Riemann theta functions and algebraic curves can be used to
compute linear matrix representations of algebraic curves. A theorem
from classical algebraic geometry states that every homogenous
polynomial $f \in P^2\CC[x_0,x_1,x_2]$ can be written in the form
\[
   f(x_0,x_1,x_2) = \text{det}
   \left( A x_0 + B x_1 + C x_2 \right)
\]
where $A,B,C$ are symmetric complex matrices which can be efficiently
computed using Riemann theta functions. Furthermore, when the polynomial
has real coefficients then $A,B,C$ are symmetric real matrices and such
representations are important in the study of spectrahedra --- the
solution spaces of semidefinite programs. \cite{PSV10}

The purpose of this research is to develop efficient and performant
algorithms for computing with Abelian functions on Riemann surfaces. The
computational tools developed in this research program have far-reaching
and varied applications.

