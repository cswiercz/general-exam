%%%%%%%%%%%%%%%%%%%%%%%%%%%%%%%%%%%%%%%%%%%%%%%%%%%%%%%%%%%%%%%%%%%%%%%%%%%%%%%
\section{Complex Algebraic Geometry --- Period Matrices and the Abel Map}
%%%%%%%%%%%%%%%%%%%%%%%%%%%%%%%%%%%%%%%%%%%%%%%%%%%%%%%%%%%%%%%%%%%%%%%%%%%%%%%

In this section we give a brief introduction to the theory of complex
algebraic curves. We explain the concept of a period matrix and the Abel
map, a particular function from a Riemann surface to the complex
numbers.

%..............................................................................
\subsection{Projective Space and Projective Curves}
%..............................................................................

We begin with an introduction to complex projective geometry. The point
of the construction is to make concrete the way in which we analyze the
behavior of functions, such as polynomials, at infinity without having
to resort to techniques separate from those used at finite points.

In fact, one with an elementary complex analysis background will have
already seen an example of projective geometry. The Riemann sphere is
the complex plane $\CC$ with a ``point at infinity'' added. If $z$ is
the coordinate in $\CC$ then $w = 1/z$ is the coordinate around the
point at infinity. This explains why, for example, the exponential
function
\[
    e^z = \sum_{n=0}^\infty z^n / n!
\]
has an essential singularity at infinity because, in the coordinate $w$
at infinity, the exponential function is expressed by the series
\[
    \sum_{n=0}^\infty w^{-n} / n!.
\]

The definition of the Riemann sphere is made explicit by the following
construction: consider the set $U = \CC^2 - \{(0,0)\}$. Define the
equivalence relation
\[
    (a_0, a_1) \sim (\lambda a_0, \lambda a_1),
    \quad \forall \lambda \neq 0.
\]
This is equivalent to saying that all ratios $a_0 : a_1$ are considered
equivalent points in $U$. The {\it complex projective line} $P^1\CC$ is
the set of all such equivalence classes. That is,
\[
    P^1\CC := \CC^2 / \sim.
\]
Points in $P^1\CC$ are typically written as $(a_0 : a_1)$.

The claim is that $P^1\CC$ is precisely the Riemann sphere. Consider the
two subsets
\begin{align*}
    U_0 &= \{ (a_0 : a_1) \in P^1\CC \; | \; a_0 \neq 0 \}, \\
    U_1 &= \{ (a_0 : a_1) \in P^1\CC \; | \; a_1 \neq 0 \}.
\end{align*}
For any $(a_0 : a_1) \in U_0$ we have, by the equivalence property,
\[
    (a_0 : a_1) = (1 : a_1/a_0) = (1 : a).
\]
Similarly, every point $(b_0 : b_1) \in U_1$ can be written as $(b : 1)$
and in the intersection $U_0 \cap U_1$ every element can be written in
either of these two forms. Each of these subspaces are isomorphic to
$\CC$ since the maps $\phi_0 : (a_0 : a_1) \in U_0 \mapsto a_1 / a_0 \in
\CC$ and $\phi_1 : (a_0 : a_1) \in U_1 \mapsto a_0 / a_1 \in \CC$ are
continuous bijections with inverses $\phi^{-1}_0 : a \in \CC \mapsto (1
: a) \in U_0$ and $\phi^{-1}_1 : b \in \CC \mapsto (b : 1) \in U_1$.

Finally, note that $(0 : 1)$ is the only projective point in $U_1$ which
is not in $U_0$. Therefore, we identify $U_0$ with the complex plane (at
$z=0$) and the point $(0 : 1)$ with the point at infinity and set
\begin{equation}
  P^1\CC = U_0 \cup \{ (0 : 1) \} \equiv \CC \cup \{(0 : 1)\}.
\end{equation}


%------------------------------------------------------------------------------
\subsection{Introduction to Algebraic Curves}
%------------------------------------------------------------------------------

Primary references. \cite{Griffiths89, Ueno97}

A complex plane algebraic curve, $C$, is the zero locus of the
homogenization of a polynomial $f \in \CC[x,y]$. That is, given a
polynomial $f(x,y) = a_n(x) y^n + a_{n-1}(x)y^{n-1} + \cdots + a_0(x)$
its homogenization is the polynomial $F \in P^2\CC[x_0,x_1,x_2]$ where
\[
  F(x_0,x_1,x_2) = x_0^d f(x_1/x_0,x_2/x_0).
\]
where $d$ is the degree of $F$ and $P^2\CC$ denotes two-dimensional
complex projective space. We often consider the {\it affine curve}
$f(x,y) = F(1,x,y)$, which can be thought of as a projection of the
curve $F$ to $\CC^2[x,y]$. Therefore, a {\it complex plane algebraic
  curve} is the set
\[
  C = \left\{
  (x_0 : x_1 : x_2) \in P^2\CC : F(x_0,x_1,x_2) = 0
  \right\}.
\]

%..............................................................................
\subsubsection*{Singular Points}
%..............................................................................

A point $p = (a_0 : a_1 : a_2) \in C$ is a {\it singular point of $C$},
or a {\it multiple point of $C$}, if
\[
  \left(
    \frac{\partial F}{\partial x_0} (p),
    \frac{\partial F}{\partial x_1} (p),
    \frac{\partial F}{\partial x_2} (p)
  \right)
  = 0.
\]
Just as with roots, singular points have a multiplicity. Without loss of
generality we consider the case when $p = (1 : 0 : 0)$ is a singular
point of $F$. The corresponding affine curve is $f(x,y) = F(1,x,y)$
which we write as
\[
  f(x,y) = \sum_{i+j \geq 2}^d a_{ij} x^iy^j.
\]
Note that the constant term is zero since we've centered the curve at
$(0, 0)$, a point of the affine curve, and the linear term vanishes
since $(0,0)$ is a singular point. We then write
\[
  f(x,y) = f_m(x,y) + f_{m+1}(x,y) + \cdots + f_d(x,y), \quad m \geq 2
\]
where each $f_n$ is the sum of all of the terms of $f$ of degree $n$;
that is, terms of the form $a_{ij}x^iy^j$ such that $i+j=n$. The
smallest such $m$ with non-zero term $f_m$ appearing in $f$ is called
the {\it multiplicity} of the singular point $(1 : 0 : 0)$. This
homogenous term can be factored into linear factors
\[
  f_m(x,y) = \prod_{j=1}^m (\alpha_jx - \beta_jy).
\]
We call the space $f_m(x,y) = 0$ the {\it tangent cone of the plane
  curve $C$} at $(1 : 0 : 0)$, which consists of a finite number of
lines.

If the singular point $p$ is elsewhere we can simply map the singular
point to $(1 : 0 : 0)$ via projective transformation. It is also
possible to deal with the case directly. If $p = (1 : a : b)$ is a
singular point then we write the affine curve in the form
\[
    f(x,y) = \sum_{i+j \geq 2}^d a_{ij} (x-a)^i(y-b)^j
\]
and then write as a sum of homogenous polynomials $g_n(x-a,y-b)$. In the
case when the singular point is $p = (0 : 1 : b)$ we repeat the above
process with the affine curve $g(u,v) = F(u,0,v)$ centered at
$(0,b)$. Finally, the case $p = (0 : 0 : 1)$ [XXX]

\begin{example}
Let us consider the cubic curve
\[
    C: F(x_0,x_1,x_2) = x_0x_1^2 + x_1^3 - x_2^2x_0 = 0.
\]
For $p = (a_0 : a_1 : a_2)$ we have
\begin{align*}
    \Delta_x^1 F(a) &=
        (a_1^2 - a_2^2)x_0 + a_1(2a_0 + 2a_1)x_2 - 2a_0a_2x_2 \\
    \Delta_x^2 F(a) &=
        4a_1x_0x_1 + 2(a_0+3a_1)x_1^2 - 2a_0x_2^2 - 4a_2x_2x_0.
\end{align*}
Note that the point $a = (1 : 0 : 0)$ satisfies $\Delta_x^1 F(a) \equiv
0$ and $\Delta_x^2 F(a) \not \equiv 0$. The tangent cone at $a$ is given
by
\[
    \Delta_x^2 F(a) = x_1^2 - x_2^2 = 0
\]
which represents two lines $x_1 - x_2 = 0$ and $x_1 + x_2 = 0$. When the
tangent cone at $a$ can be completely factored into linear terms we say
$a$ is an {\it ordinary} double point of the curve, $C$. On the affine
plane $x_0=1$, equivalent to setting $x = x_1/x_0$ and $y=x_2/x_0$, this
gives the curve $f(x,y) = y^3 + xy^2 - x$. In Figure \ref{fig:
  double-pt-example} we see the double-point singularity appearing at
the origin where two ``branches'' of the curve meet.
\end{example}

\begin{figure}
  \label{fig: double-pt-example}
  \caption{A real plot of the curve $f(x,y) = y^3 + xy^2 - x$ in the $x-y$ plane.}
\end{figure}

%------------------------------------------------------------------------------
\subsection{Connection to Riemann Surfaces}
%------------------------------------------------------------------------------

There exists a close relationship between the study of compact Riemann
surfaces and that of algebraic curves. A Riemann surface $\tilde{C}$ is
simply a complex manifold of complex dimension one. This relationship is
embodied by the following two theorems.

\begin{theorem} \label{thm: normalization}
  {\it (Normalization Theorem.)} For any irreducible algebraic curve $C
  \subset P^2\CC$ there exists a compact Riemann surface $\tilde{C}$ and
  a holomorphic mapping
  \[
    \sigma : \tilde{C} \to P^2\CC
  \]
  such that $\sigma( \tilde{C} ) = C$ adn $\sigma$ is injective on the
  inverse image of the set of smooth points of $C$.
\end{theorem}

A Riemann surface together with the mapping $\sigma$ is called the {\it
  normalization of $C$}. Loosely speaking, the normalization theorem
states that an algebraic curve is a Riemann surface except at the
singular points.

%% In particular, any irreducible plane algebraic curves admits a
%% holomorphic parametric representation and the domain of definition of
%% this representation is a compact Riemann surface [[[]]]. The following
%% two theorems indeed show
