%%%%%%%%%%%%%%%%%%%%%%%%%%%%%%%%%%%%%%%%%%%%%%%%%%%%%%%%%%%%%%%%%%%%%%%%%%%%%%%
\section{Complex Algebraic Geometry --- Period Matrices and the Abel Map}
%%%%%%%%%%%%%%%%%%%%%%%%%%%%%%%%%%%%%%%%%%%%%%%%%%%%%%%%%%%%%%%%%%%%%%%%%%%%%%%

In this section we give a brief introduction to the theory of complex
algebraic curves. Primary references are \cite{Ueno97,Griffiths89}.

%..............................................................................
\subsection{The Projective Line}
%..............................................................................

We begin with an introduction to complex projective geometry. The
primary motivation behind projective geometry is to make concrete the
way in which we analyze the behavior of functions, such as polynomials,
at infinity without having to resort to techniques separate from those
used at finite points. For example, in applications we may need to
integrate a differential along a path on an algebraic curve going to
infinity. Knowing the geometry of the curve at infinity makes such an
operation computationally feasible.

In fact, one with an elementary complex analysis background will have
already seen an example of projective geometry. The Riemann sphere is
the complex plane $\CC$ with a ``point at infinity'' added. Let $z$
denote the coordinate in $\CC$ (i.e. the point $z=0$ represents the
``center'' of the complex plane). In order to talk about the point at
infinity we introduce the coordinate $w = 1/z$. That is, the analysis of
some function at $z$ ``='' $\infty$ is equivalent to transforming the
problem to the coordinate $w$ and examining its behavior at $w=0$. This
explains why, for example, the exponential function
\[
    e^z = \sum_{n=0}^\infty z^n / n!,
\]
though entire in the complex plane, has an essential singularity on the
Riemann sphere since the exponential function in the coordinate $w$
centered at $w=0$ is expressed by the series
\[
    \sum_{n=0}^\infty \frac{w^{-n}}{n!}
    =
    \sum_{m=-\infty}^0 \frac{w^m}{(-m)!}.
\]

This point at infinity is not rigorously defined for it doesn't make
sense to {\it set} $z=\infty$. The definition of the Riemann sphere is
made explicit by the following construction: consider the set $U = \CC^2
- \{(0,0)\}$. Define the equivalence relation
\[
    (a_0, a_1) \sim (\lambda a_0, \lambda a_1),
    \quad \forall \lambda \in \CC - \{0\}.
\]
That is that two points $(a_0,a_1)$ and $(b_0,b_1)$ in $U$ are
considered the same point if the ratios $a_0 : a_1$ and $b_0 : b_1$ are
equal. The set of all points $(b_0,b_1)$ equal to $(a_0,a_1)$ is called
the {\it equivalence class} of $(a_0,a_1)$ and the {\it complex
  projective line} $P^1\CC$ is the set of all such equivalence
classes. That is,
\[
    P^1\CC := \CC^2 / \sim.
\]
The equivalence class of $(a_0,a_1)$, called a ``point'' in $P^1\CC$, is
written $(a_0 : a_1) \in P^1\CC$.

The claim is that $P^1\CC$ is precisely the Riemann sphere. Consider the
two subsets
\begin{align*}
    U_0 &= \{ (a_0 : a_1) \in P^1\CC \; | \; a_0 \neq 0 \}, \\
    U_1 &= \{ (a_0 : a_1) \in P^1\CC \; | \; a_1 \neq 0 \}.
\end{align*}
For any $(a_0 : a_1) \in U_0$ we have, by the equivalence property,
\[
    (a_0 : a_1) = (1 : a_1/a_0) = (1 : a).
\]
Similarly, every point $(b_0 : b_1) \in U_1$ can be written as $(b : 1)$
and every point in the intersection $U_0 \cap U_1$ can be written in
either of these two forms. Each of these subspaces are isomorphic to
$\CC$ since the maps
\begin{align*}
  \phi_0 : U_0 \to \CC,
  \quad
  \phi_0 \left( (a_0 : a_1) \right) = a_1 / a_0
  & \quad \text{ and } \\
  \phi_1 : U_1 \to \CC,
  \quad
  \phi_1 \left( (a_0 : a_1) \right) = a_0 / a_1 &
\end{align*}
are continuous bijections with inverses
\begin{gather}
  \phi^{-1}_0(a) = (1 : a) \\
  \phi^{-1}_1(b) = (b : 1).
\end{gather}

Finally, note that $(0 : 1)$ is the only projective point in $U_1$ which
is not in $U_0$. Therefore, we identify $U_0$ with the complex plane (in
the coordinate $z$) and the point $P_\infty = (0 : 1)$ with the point at
infinity and set
\begin{equation} \label{eqn: projective-line}
  P^1\CC = U_0 \cup \{ (0 : 1) \} \cong \CC \cup P_\infty.
\end{equation}

$P_\infty$ is indeed considered the point at infinity on the Riemann
sphere for if one were to consider the image of $(0 : 1)$ under
$\phi_0$, though undefined since it doesn't belong to $U_0$, it would
map to $z = 1 / 0$ ``='' $\infty$. Again, this does not make sense
without the complex projective space construction above but is merely
used to illustrate the point. The coordinate transformation from $z$ to
$w$ at the beginning of this section is equivalent to identifying $U_1$
with the complex plane $\CC$ and $\{(1 : 0)\}$ with the point at
infinity, instead. (Which now corresponds to $z=0$.)

%------------------------------------------------------------------------------
\subsection{The Projective Plane}
%------------------------------------------------------------------------------

The natural environment we will use in the sequel is not the complex
projective line but the complex projective plane. In this section we
will construct the projective plane and examine its geometric
properties. The construction is similar to that of the projective line.

Let $U = \CC^3 - \{(0,0,0)\}$. Following the strategy of the previous
section, consider the set of all ratios $(a_0 : a_1 : a_2)$, that is,
the collection of all equivalence classes under the equivalence relation
$(a_0 : a_1 : a_2) \sim (\lambda a_0 : \lambda a_1 : \lambda a_2),
\forall \lambda \in \CC - \{0\}$. This space of all such equivalence
classes is called the two-dimensional complex projective space or {\it
  the projective plane} and is denoted $P^2\CC$.

Define the subsets $U_0,U_1,U_2$ by
\[
    U_j = \{ (a_0 : a_1 : a_2) \in P^2\CC \; | \; a_j \neq 0 \}
\]
and note that all $(a_0 : a_1 : a_2) \in U_0$ satisfied $(a_0 : a_1 :
a_2) = (1 : a_1/a_0 : a_2/a_0)$. We define the bijective mapping
\begin{gather*}
    \phi_0 : U_0 \to \CC^2 \\
    \phi_0 \left( (a_0 : a_1 : a_2) \right)
    =
    \left( \frac{a_1}{a_0}, \frac{a_2}{a_0} \right), \\
    \phi_0^{-1} \left( (x,y) \right)
    =
    (1 : x : y).
\end{gather*}
The mappings $\phi_1$ and $\phi_2$ are similarly defined on $U_1$ and
$U_2$, respectively. Therefore, we can identify $U_0$ with the complex
plane $\CC^2$.

Consider the space $U_0^c = P^2\CC - U_0$. By definition, every point in
$U_0^c$ is of the form $(0 : a_1 : a_2)$. Therefore, every point in
$U_0^c$ determines a point on the complex projective line $P^1\CC$. The
converse is true as well, resulting in the bijection
\[
    (0 : a_1 : a_2) \in P^2\CC
    \; \leftrightarrow \;
    (a_1 : a_2) \in P^1\CC.
\]
By identifying $U_0^c$ with $P^1\CC$ we may write
\begin{equation} \label{eqn: proejctive-plane}
  P^2\CC = U_0 \cup U_0^c \cong \CC^2 \cup P^1\CC
\end{equation}
where $U_0^c \cong P^1\CC$ is called the {\it line at infinity}, denoted
$l_\infty$, and $U_0 \cong \CC^2$ is called the {\it complex affine
  plane}. Note that we may instead identify the complex affine plane
with the spaces $U_1$ or $U_2$ and the line at infinity with their
complements.

We saw a natural geometric interpretation of $P^1\CC$ in the previous
section. Does such an interpretation exist for $P^2\CC$? Consider a line
in the complex affine plane $\CC^2$ which can be written in the form
\[
    \alpha + \beta x + \gamma y = 0,
    \quad \text{where }
    (\beta,\gamma) \neq 0, \alpha, \beta, \gamma \in \CC.
\]
Using the inverse mapping $\phi_0^{-1}$ on $\CC^2$ we have
\[
    x = \frac{x_1}{x_0} \text{ and } y = \frac{x_2}{x_0},
\]
where $(x_0 : x_1 : x_2)$ are the coordinates of $P^2\CC$, and we get
the line
\[
    \alpha x_0 + \beta x_1 + \gamma x_2 = 0.
\]
This equation, called the {\it homogenization} of the affine curve,
makes sense in all of $P^2\CC$. Setting $x_0 = 1$ gives us the original
affine line. On the other hand, setting $x_0 = 0$ gives us the equation
\[
    \beta x_1 + \gamma x_2 = 0
\]
which is the equation of the line in $l_\infty$. However, this implies
$x_1 / x_2 = - \gamma / \beta$. Hence the projective point $(0 : -\gamma
: \beta)$ satisfies the equation
\[
    \alpha x_0 + \beta x_1 + \gamma x_2 = 0
\]
and, in fact, is the only projective point in $l_\infty$ on the line.

This means that the line ``intersects'' $l_\infty$ at the point $(0 :
-\gamma : \beta)$ and that this intersection point is only dependent on
the slope of the affine portion of the line and is independent of the
parameter $\alpha$. Hence, the line at infinity has the geometric
meaning that each point on it is the intersection point of an entire
family of parallel lines in $\CC^2$. This leads to a generalization of a
theorem from classical planar geometry: {\it any two, distinct lines in
  $P^2\CC$ intersect at exactly one point}.

%------------------------------------------------------------------------------
\subsection{Projective Plane Curves}
%------------------------------------------------------------------------------

The set of all points $(x_0, x_1, x_2)$ satisfying
\[
    \alpha x_0 + \beta x_1 + \gamma x_2 = 0
\]
is called a projective line and is a simple example of a projective
algebraic curve (of degree one). In this section we will introduce
various properties of general projective curves.

An affine complex plane algebraic curve is the zero locus of the
homogenization of a polynomial $f \in \CC[x,y]$. That is, given a
polynomial $f(x,y) = \alpha_n(x) y^n + \alpha_{n-1}(x)y^{n-1} + \cdots +
\alpha_0(x)$ its homogenization is the polynomial $F \in
P^2\CC[x_0,x_1,x_2]$ where
\[
    F(x_0,x_1,x_2) = x_0^d f(x_1/x_0,x_2/x_0).
\]
where $d$ is the degree of $F$. The homogeneity of $F$ means that we can
always write
\[
    F(x_0,x_1,x_2) = \sum_{i+j+k=d} \alpha_{ijk} x_0^i x_1^j x_2^k.
\]

In terms of the projective polynomial $F$, its affine part can be
written $f(x,y) = F(1,x,y)$. As in the case of a projective line, $f$
can be thought of as a projection of the polynomial $F$ onto $\CC^2$ and
there is always a one-to-one correspondence between an affine polynomial
and its homogenization. Therefore, a {\it complex plane algebraic curve}
is the set
\[
  C = \left\{
  (x_0 : x_1 : x_2) \in P^2\CC : F(x_0,x_1,x_2) = 0
  \right\}.
\]
also commonly denoted
\[
    C: F(x_0,x_1,x_2) = 0.
\]

%% %..............................................................................
%% \subsubsection*{Singular Points}
%% %..............................................................................

Important to the study of projective curves, and specifically in the
computational work described here, are the notions of singular points. A
point $a = (a_0 : a_1 : a_2) \in C$ is a {\it singular point of $C$}, or
a {\it multiple point of $C$}, if
\[
  \left(
    \frac{\partial F}{\partial x_0} (a),
    \frac{\partial F}{\partial x_1} (a),
    \frac{\partial F}{\partial x_2} (a)
  \right)
  = (0,0,0).
\]
Singular points admit a multiplicity. To begin, consider the case when
$a = (1 : 0 : 0)$ (the point $(0,0)$ in the affine plane $\CC^2$) is a
singular point of $F$. Write the corresponding affine curve $f$ as
\[
  f(x,y) = \sum_{i+j \geq 2}^d c_{ij} x^iy^j.
\]
Note that the constant term is zero since we've centered the curve at
$(0, 0)$, a point of the affine curve, and the linear term vanishes
since $(0,0)$ is a singular point. We then write
\[
  f(x,y) = f_m(x,y) + f_{m+1}(x,y) + \cdots + f_d(x,y), \quad m \geq 2
\]
where each $f_n$ is the sum of all of the terms of $f$ of degree $n$;
that is, terms of the form $c_{ij}x^iy^j$ such that $i+j=n$. The
smallest such $m$ with non-zero term $f_m$ appearing in $f$ is called
the {\it multiplicity} of the singular point $(1 : 0 :
0)$. Singularities with multiplicity two are called {\it double points}
since play a special role in the theory of Riemann surfaces.

The homogeneous term $f_m$ can be factored into linear factors
\[
  f_m(x,y) = \prod_{j=1}^m (\alpha_j x - \beta_j y).
\]
We call the space $f_m(x,y) = 0$ the {\it tangent cone of the plane
  curve $C$} at $(1 : 0 : 0)$, which consists of a finite number of
intersecting lines $L_j : \alpha_j x - \beta_j y$.

If the singular point $p$ is elsewhere we can simply map it to the point
$(1 : 0 : 0)$ via projective transformations. [{\it Note: I guess I
    could talk about these since its what I do in my code.}] It is also
possible to deal with the case directly. If $p = (1 : a : b)$ is a
singular point then we write the affine curve in the form
\[
    f(x,y) = \sum_{i+j \geq 2}^d a_{ij} (x-a)^i(y-b)^j
\]
and then write as a sum of homogeneous polynomials $g_n(x-a,y-b)$. In the
case when the singular point is $p = (0 : 1 : b)$ we repeat the above
process with the affine curve $g(u,v) = F(u,0,v)$ centered at
$(0,b)$. Finally, the case $p = (0 : 0 : 1)$

\begin{example} \label{ex: cubic}
Let us consider the cubic curve
\[
    C: F(x_0,x_1,x_2) =
    x_0^4 x_2^3 + 2 x_0^3 x_1^3 x_2 - x_1^7
\]
In complex affine space $x_0 = 1$ this curve is
\[
    f(x,y) = F(1,x,y) = y^3 + 2 x^3 y - x^7
\]
A plot of $f$ for $x,y$ real is shown in Figure \ref{fig: example-cubic}.
For $a = (a_0 : a_1 : a_2)$ we have
\begin{align} \label{eq: singular-conditions}
    \frac{\partial F}{\partial x_0}(a)
    &=
    4 a_{0}^{3} a_{2}^{3} + 6 a_{0}^{2} a_{1}^{3} a_{2} \notag \\
    \frac{\partial F}{\partial x_1}(a)
    &=
    6 a_{0}^{3} a_{1}^{2} a_{2} - 7 a_{1}^{6} \notag \\
    \frac{\partial F}{\partial x_2}(a)
    &=
    3 a_{0}^{4} a_{2}^{2} + 2 a_{0}^{3} a_{1}^{3}
\end{align}

\begin{figure}
  \label{fig: example-cubic}
  \centering
  \includegraphics[width=0.8\textwidth]{images/2_example_curve.png}
  \caption{A real plot of the curve $f(x,y) = y^3 + 2 x^3 y - x^7$.}
\end{figure}

First, we find the finite singular points of $C$. Setting $a_0=1$ and
solving the above equations for $a_1$ and $a_2$ we see that $p = (1 : 0
: 0)$ is the only finite singular point of $F$. Note
\begin{gather*}
    f(x,y) = f_3(x,y) + f_4(x,y) + f_7(x,y) \\
    f_3(x,y) = y^3, \; \;
    f_4(x,y) = 2 x^3 y, \; \;
    f_7(x,y) = -x^7,
\end{gather*}
and that $f_3$, $f_4$, and $f_7$ are homogeneous of degrees 3, 4, and 7,
respectively. Therefore, $p$ is a singular point of multiplicity three
with
\[
    f_3(x,y) = y^3 = 0
\]
as the equation for the tangent cone at $p$. These properties are
evident from Figure \ref{fig: example-cubic} where, near the point $p$,
the curve looks like the intersection of three curves well approximated
by the line $y = 0$.

Setting $a_0 = 0$, the only expression in Equation \eqref{eq:
  singular-conditions} that doesn't reduce to zero is
\[
    \frac{\partial F}{\partial x_1}(0,a_1,a_2))
    =
    - 7 a_{1}^{6} = 0
\]
implying that the point $a = (0 : 0 : 1)$ is the only singular point at
infinity. The curve at infinity centered at $(0 : 0 : 1)$ is
\[
    g(u,v) = F(u,v,1) = u^{4} + 2 u^{3} v^{3} - v^{7}
\]
The order of this singularity is four since this is the degree of the
lowest degree homogeneous term. The tangent cone at $a$ is $g_4(u,v) =
u^4$.
%% Figure \ref{fig: example-cubic} suggests this is the case the curve
%% at the point $(u,v) = (0,0)$.
\end{example}


%------------------------------------------------------------------------------
\subsection{Connection to Riemann Surfaces}
%------------------------------------------------------------------------------

There is a close relationship between the study of compact Riemann
surfaces and that of algebraic curves. Recall that a Riemann surface
$\tilde{C}$ is a complex manifold of complex dimension one endowed with
an {\it atlas}: an open covering $\{U_\alpha \}_{\alpha \in A}$ of
$\tilde{C}$ together with a collection of homeomorphisms $\{z_\alpha :
U_\alpha \to \CC\}_{\alpha \in A}$, called {\it local parameters}, such
that every pair of {\it transition functions}
\[
    f_{\beta,\alpha} := z_\beta \circ z_\alpha^{-1} :
    z_\alpha \left( U_\alpha \cap U_\beta \right)
    \to
    z_\beta \left( U_\alpha \cap U_\beta \right)
\]
is holomorphic. The pairs $(U_\alpha, z_\alpha)$ are called {\it
  coordinate charts}. In other words, a Riemann surface is a topological
space such that for all points $P \in \tilde{C}$ there is a neighborhood
of $P$ homeomorphic to an open subset of the complex plane and one can
analytically continue along the surface. (Via transition functions.)

The Riemann sphere $\tilde{C} = \CC^*$ is an example of a Riemann
surface. Its atlas consists of two coordinate charts $(U_1, z_1)$ and
$(U_2, z_2)$ with
\begin{align*}
    U_1 = \CC, & \quad z_1 = z, \\
    U_2 = \left( \CC - \{0\} \right) \cup \{ \infty \}, & \quad z_2 = 1/z.
\end{align*}
This is a valid atlas for the transition functions
\begin{gather*}
    f_{1,2}, f_{2,1} : \CC - \{0\} \to \CC - \{0\} \\
    f_{1,2} = z_1 \circ z_2^{-1} = 1/z \\
    f_{2,1} = z_2 \circ z_1^{-1} = 1/z
\end{gather*}
are holomorphic on $U_1 \cap U_2 = \CC - \{0\}$.

This relationship between curves and Riemann surfaces are embodied by
the following two theorems.

\begin{theorem} \label{thm: normalization}
  {\it (Normalization Theorem.)} For any irreducible algebraic curve $C
  \subset P^2\CC$ there exists a compact Riemann surface $\tilde{C}$ and
  a holomorphic mapping
  \[
      \sigma : \tilde{C} \to P^2\CC
  \]
  such that $\sigma( \tilde{C} ) = C$ and $\sigma$ is injective on the
  inverse image of the set of smooth points of $C$.
\end{theorem}

A Riemann surface together with the mapping $\sigma$ is called the {\it
  normalization of $C$}. Loosely speaking, the normalization theorem
states that an algebraic curve is a Riemann surface except at the
singular points.

Finally, every compact Riemann surface can be represented by an
algebraic curve.

\begin{theorem} \label{thm: repr-theorem}
  Any compact Riemann surface $\tilde{C}$ can be obtained through the
  normalization of a certain plane algebraic curve $C$ with at most
  ordinary double points. That is, there exists a holomorphic mapping
  \[
      \sigma : \tilde{C} \to P^2\CC
  \]
  such that $\sigma(\tilde{C})$ is an algebraic curve possessing at most
  ordinary double points.
\end{theorem}
