\documentclass{article}

\usepackage{amsmath,amssymb,amsthm}
\usepackage{algorithm, algpseudocode}
\renewcommand{\algorithmicrequire}{\textbf{Input:}\,}
\renewcommand{\algorithmicensure}{\textbf{Output:}}
\renewcommand{\algorithmicforall}{\textbf{for each}}
\newcommand{\Input}{\Require}
\newcommand{\Output}{\Ensure}
\newcommand{\ForEach}{\ForAll}

%
% Custom operator declarations
%
\DeclareMathOperator{\ZZ}{\mathbb{Z}}
\DeclareMathOperator{\CC}{\mathbb{C}}
\DeclareMathOperator{\KK}{\mathbb{C}}
\DeclareMathOperator{\PP}{\mathbb{P}}
\DeclareMathOperator{\hh}{\mathfrak{h}}
\DeclareMathOperator{\re}{\text{Re}}
\DeclareMathOperator{\im}{\text{Im}}
\newcommand{\thetachar}[2] {\theta {\scriptsize \begin{bmatrix}#1\\#2\end{bmatrix}}}
\newcommand{\thetacharsm}[2] {\theta \left[ \begin{smallmatrix} #1
      \\ #2 \end{smallmatrix} \right]}

\title{General Examination -- Algebro-Geometric
  Approach to Nonlinear Integrable Equations}

\author{
Chris Swierczewski\\
University of Washington\\
Department of Applied Mathematics}
\date{\today}

%%%%%%%%%%%%%%%%%%%%%%%%%%%%%%%%%%%%%%%%%%%%%%%%%%%%%%%%%%%%%%%%%%%%%%%%%%%%%%%
\begin{document}
%%%%%%%%%%%%%%%%%%%%%%%%%%%%%%%%%%%%%%%%%%%%%%%%%%%%%%%%%%%%%%%%%%%%%%%%%%%%%%%

\maketitle

%%%%%%%%%%%%%%%%%%%%%%%%%%%%%%%%%%%%%%%%%%%%%%%%%%%%%%%%%%%%%%%%%%%%%%%%%%%%%%%
\section{Introduction}
%%%%%%%%%%%%%%%%%%%%%%%%%%%%%%%%%%%%%%%%%%%%%%%%%%%%%%%%%%%%%%%%%%%%%%%%%%%%%%%

The Kadomtsev-Petviashvili (KP) equation is a partial differential
equation used to describe the surface height of a two-dimensional
periodic shallow water wave. Depending on certain physical
considerations [X], which we will ignore, one can derive either of
the following two equations
\begin{align}
  \left(-4u_t + 6uu_x + u_{xxx}\right)_x + 3\sigma^2 u_{yy} = 0, \quad
  \sigma^2 = -1, \label{eqn: KP1} \\
  \left(-4u_t + 6uu_x + u_{xxx}\right)_x + 3\sigma^2 u_{yy} = 0, \quad
  \sigma^2 = +1. \label{eqn: KP2}
\end{align}
where $u(x,y,t)$ is the surface height as a function of position $(x,y)$
and time $t$. In the sequel we will not rely on this distinction and
simply refer to the ``KP equation''.

The KP equation admits a large family of quasiperiodic solutions of the
form
\begin{equation} \label{eqn: KPsoln}
  u(x,y,t) = 2 \partial_x^2 \log \theta(Ux+Vy+Wt+z_0, \Omega)
\end{equation}
where $\theta : \CC^g \times \hh_g \to \mathbb{C}$ is the Riemann theta
function acting on the space of $g$-dimensional complex vectors and $g
\times g$ {\it ``Riemann matrices''} --- complex symmetric matrices
with positive definite imaginary part.
\begin{equation} \label{eqn: RiemannTheta}
  \theta(z,\Omega) = \sum_{n \in \ZZ^g}
  e^{2 \pi i \left( \tfrac{1}{2} n \cdot \Omega n + n \cdot z \right)}
\end{equation}
These solutions are the so-called ``theta function solutions'' to the KP
equation and families of such solutions exist for every $g > 0$. In
fact, the totality of solutions of this form are dense the space of all
periodic solutions to KP [X]. The constants $U,V,W,z_0 \in \CC^g$ and
$\Omega \in \hh_g$ in \eqref{eqn: KPsoln} are all determined from a
Riemann surface corresponding to a complex plane algebraic curve and any
such curve can produce a solution to KP [X].

The purpose of this research is to


%%%%%%%%%%%%%%%%%%%%%%%%%%%%%%%%%%%%%%%%%%%%%%%%%%%%%%%%%%%%%%%%%%%%%%%%%%%%%%%
\section{Complex Algebraic Geometry --- Period Matrices and the Abel Map}
%%%%%%%%%%%%%%%%%%%%%%%%%%%%%%%%%%%%%%%%%%%%%%%%%%%%%%%%%%%%%%%%%%%%%%%%%%%%%%%

In this section we give a brief introduction to the theory of complex
algebraic curves. We explain the concept of a period matrix and the Abel
map, a particular function from a Riemann surface to the complex
numbers.

%------------------------------------------------------------------------------
\subsection{Algebraic Curves and Riemann Surfaces}
%------------------------------------------------------------------------------

A complex plane algebraic curve, $C$, is the zero locus of the
homogenization of a polynomial $f \in \CC[\lambda,\mu]$. That is, given
a polynomial $f(\lambda,\mu) = a_n(\lambda) \mu^n + a_{n-1}\mu^{n-1} +
\cdots + a_0(\lambda)$ its homogenization is the polynomial $F \in
P_2\CC[\zeta,\xi,\eta]$ where
\[
f(\lambda,\mu) = F(1,\lambda,\mu)
\quad \text{and} \quad
F(\zeta,\xi,\eta) = \zeta^n f(\lambda/\zeta,\mu/\zeta).
\]
($P_2\CC$ denotes two-dimensional complex projective space.) Therefore,
a complex plane algebraic curve is the set
\[
C = \left\{
(\zeta,\xi,\eta) \in P^2\CC : F(\zeta,\xi,\eta) = 0
\right\}.
\]

There exists a close relationship between the study of compact Riemann
surfaces and that of algebraic curves. A Riemann surface $\tilde{C}$ is
simply a complex manifold of complex dimension one. An algebraic curve
$C$ is the set

where $F \in P^2\CC[\zeta,\xi,\eta]$ is the homogenization of a plane
polynomial $f \in \CC[\lambda,\mu]$. By definition, 

In particular, any irreducible plane algebraic curves admits a
holomorphic parametric representation and the domain of definition of
this representation is a compact Riemann surface [[[]]]. The following
two theorems indeed show

We will postpone precisely how to determine these constants until more
machinery is developed in the following chapter.


%%%%%%%%%%%%%%%%%%%%%%%%%%%%%%%%%%%%%%%%%%%%%%%%%%%%%%%%%%%%%%%%%%%%%%%%%%%%%%%
\section{Computing Period Matrices of Algebraic Curves and the Abel Map}
%%%%%%%%%%%%%%%%%%%%%%%%%%%%%%%%%%%%%%%%%%%%%%%%%%%%%%%%%%%%%%%%%%%%%%%%%%%%%%%

In this section we will look at algorithmically computing period
matrices. Each subsection here examines in close detail a major
component of the algorithm and provides a theoretical overview of the
component, an algorithm for computing the component, and an example
presented in {\tt abelfunctions}, a Python software package


%------------------------------------------------------------------------------
\subsection{abelfunctions}
%------------------------------------------------------------------------------

%------------------------------------------------------------------------------
\subsection{Puiseux Series}
%------------------------------------------------------------------------------

%
\subsubsection{Theory}
%

%
\subsubsection{Algorithm}
%

\begin{algorithm}[h]
\caption{POLYGON --- Returns the Newton polygon of the polynomial $F =
  F(X,Y)$.}
\label{alg: puiseux-polygon}
\begin{algorithmic}[1]
\Input

$F,X,Y$ --- A polynomial.

$I$ --- If $I=0$, return the Newton polygon. If $I=1$, return only the
segments of the Newton polygon with negative slope. $I=1$ is used to
compute terms beyond the first of the Puiseux series.

\Output A set of lists $(q,m,l,\Phi)$ where
\begin{itemize}
  \item $q,m,l$ defines a line segment $\Delta: qj+mi=l$ in the $(i,j)$
    plane,
  \item $\Phi(Z) = \sum_{(i,j)\in\Delta} a_{ij} Z^{(i-i_0)/q} \in
    \CC[Z]$ where $i_0$ is the smallest value of $i$ such that there is
    a point $(i_0,j)\in\Delta$.
\end{itemize}

\Function{POLYGON}{$F,X,Y,I$}
  \State
\EndFunction
\end{algorithmic}
\end{algorithm}


\begin{algorithm}[h]
\caption{REGULAR --- Given the branching or singular part of a Puiseux
  series, computes the regular part of the series.}
\label{alg: puiseux-regular}
\begin{algorithmic}[1]
\Input

$S$ --- a finite set of pairs $\{(\pi_k,F_k)\}$

$X,Y$ --- the dependent and independent variables, respectively

$H$ --- bound on the number of desired terms of the series

\Output $R$ --- a finite set of pairs $\{(\pi_k,F_k)\}$ containing at
least $H$ terms

\Function{Regular}{$S,X,Y,H$}
  \State $R \leftarrow ()$
  \ForEach{$(\pi,F)$ {\bf in} $S$}
    \While{$\text{len}(\pi) < H$}

      \State $m \leftarrow \text{min} \{ j \; /$
      \Call{COEFFICIENT}{$F,0,j$} $ \; | \; j \neq 0 \}$

      \State $\beta \leftarrow$ \Call{COEFFICIENT}{$F,0,m$} /
      \Call{COEFFICIENT}{$F,1,0$}

      \State $\tau \leftarrow (1,1,m,\beta)$
      \State $\pi \leftarrow \pi \cup \{\tau\}$
      \State $F \leftarrow$ \Call{NEWPOLYNOMIAL}{$F,\tau,m$}
    \EndWhile
    \State $R \leftarrow R \cup \{\pi\}$
  \EndFor
\EndFunction
\end{algorithmic}
\end{algorithm}


\subsubsection{Examples}
%

%------------------------------------------------------------------------------
\subsection{Singularities}
%------------------------------------------------------------------------------

%
\subsubsection{Theory}
%
%
\subsubsection{Algorithm}
%
%
\subsubsection{Examples}
%

%------------------------------------------------------------------------------
\subsection{Holomorphic Differentials}
%------------------------------------------------------------------------------

%
\subsubsection{Theory}
%



%
\subsubsection{Algorithm}
%
%
\subsubsection{Examples}
%

%------------------------------------------------------------------------------
\subsection{Analytic Continuation}
%------------------------------------------------------------------------------

%
\subsubsection{Theory}
%

A path on a Riemann surface is a continuous map $\gamma : [0,1] \to C
\subset \CC^2$. That is, if $\gamma(t) = (x_\gamma(t), y_\gamma(t))$
then $f(x(t),y(t)) = 0$ for all $t \in [0,1]$.


%
\subsubsection{Algorithm}
%

Since analytic continuation is repeatedly performed when computing
period matrices it is important to make it performant.

Often, a path in $\CC_x$, $x_\gamma$, will be specified, such as via the
{\sc monodromy() or homology()} functions, and the corresponding
$y$-values lying above need to be determined. A naive approach to doing
so is to use a numerical root finder.

An alternate approach, which is the one used here, is to use Newton's
method --- given $x_\gamma(t_k)$ and some $y_{\gamma,j}(t_k)$ lying
above determine $y_{\gamma,j}(t_{k+1})$ above $x_\gamma(t_{k+1})$.

%
\subsubsection{Examples}
%

%------------------------------------------------------------------------------
\subsection{Monodromy}
%------------------------------------------------------------------------------

%
\subsubsection{Theory}
%
%
\subsubsection{Algorithm}
%
%
\subsubsection{Examples}
%

%------------------------------------------------------------------------------
\subsection{Homology}
%------------------------------------------------------------------------------

%
\subsubsection{Theory}
%
%
\subsubsection{Algorithm}
%
%
\subsubsection{Examples}
%

%------------------------------------------------------------------------------
\subsection{Period Matrices}
%------------------------------------------------------------------------------

%
\subsubsection{Theory}
%
%
\subsubsection{Algorithm}
%
%
\subsubsection{Examples}
%

%------------------------------------------------------------------------------
\subsection{The Abel Map}
%------------------------------------------------------------------------------

%
\subsubsection{Theory}
%
%
\subsubsection{Algorithm}
%
%
\subsubsection{Examples}
%

%%%%%%%%%%%%%%%%%%%%%%%%%%%%%%%%%%%%%%%%%%%%%%%%%%%%%%%%%%%%%%%%%%%%%%%%%%%%%%%
\section{Periodic Solutions to Nonlinear Integrable Equations}
%%%%%%%%%%%%%%%%%%%%%%%%%%%%%%%%%%%%%%%%%%%%%%%%%%%%%%%%%%%%%%%%%%%%%%%%%%%%%%%

%------------------------------------------------------------------------------
\subsection{Example: The Kadomtsev--Petviashvili Equation}
%------------------------------------------------------------------------------

%------------------------------------------------------------------------------
\subsection{Geometry of Integrable Equations}
%------------------------------------------------------------------------------


%%%%%%%%%%%%%%%%%%%%%%%%%%%%%%%%%%%%%%%%%%%%%%%%%%%%%%%%%%%%%%%%%%%%%%%%%%%%%%%
\section{Future Work}
%%%%%%%%%%%%%%%%%%%%%%%%%%%%%%%%%%%%%%%%%%%%%%%%%%%%%%%%%%%%%%%%%%%%%%%%%%%%%%%

%------------------------------------------------------------------------------
\subsection{Periodic Solutions of Integrable Equations}
%------------------------------------------------------------------------------

%------------------------------------------------------------------------------
\subsection{Fay's Prime Form}
%------------------------------------------------------------------------------

%------------------------------------------------------------------------------
\subsection{The Constructive Schottky Problem}
%------------------------------------------------------------------------------


%%%%%%%%%%%%%%%%%%%%%%%%%%%%%%%%%%%%%%%%%%%%%%%%%%%%%%%%%%%%%%%%%%%%%%%%%%%%%%%
\end{document}
%%%%%%%%%%%%%%%%%%%%%%%%%%%%%%%%%%%%%%%%%%%%%%%%%%%%%%%%%%%%%%%%%%%%%%%%%%%%%%%
