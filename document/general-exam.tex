\documentclass[10pt,twoside]{article}

%
% Use droid mono for fixed-width font
%
\usepackage[defaultmono,scale=0.8]{droidmono}
\usepackage{amsmath,amssymb,amsthm}
\usepackage[subsection]{algorithm}
\usepackage{algpseudocode}
\renewcommand{\algorithmicrequire}{\textbf{Input:}\,}
\renewcommand{\algorithmicensure}{\textbf{Output:}}
\renewcommand{\algorithmicforall}{\textbf{for each}}
\newcommand{\Input}{\Require}
\newcommand{\Output}{\Ensure}
\newcommand{\ForEach}{\ForAll}
\usepackage{graphicx}
\usepackage{caption}
\usepackage{listings}
\lstset{
%  aboveskip=\bigskipamount,
  basicstyle=\ttfamily\small,
%  belowskip=\bigskipamount,
  frame=single,
  language=Python,
  numbers=left,
  numberstyle=\tiny,
  showstringspaces=false,
  xleftmargin=4pt,
  xrightmargin=4pt
}
\lstnewenvironment{pyoutput}[1][]
{\lstset{aboveskip=-\medskipamount,numbers=none,#1}}
{}



%
% Theorems, Definitions, and Example
%
\theoremstyle{plain}
\newtheorem{theorem}{Theorem}[section]
\newtheorem{lemma}[theorem]{Lemma}
\newtheorem{proposition}[theorem]{Proposition}

\theoremstyle{definition}
\newtheorem{definition}[theorem]{Definition}
\newtheorem{conjecture}[theorem]{Conjecture}
\newtheorem{example}[theorem]{Example}

\numberwithin{equation}{section}
\numberwithin{figure}{section}


%
% Custom operator declarations
%
\DeclareMathOperator{\ZZ}{\mathbb{Z}}
\DeclareMathOperator{\RR}{\mathbb{R}}
\DeclareMathOperator{\CC}{\mathbb{C}}
\DeclareMathOperator{\KK}{\mathbb{C}}
\DeclareMathOperator{\hh}{\mathfrak{h}}
\DeclareMathOperator{\re}{\text{Re}}
\DeclareMathOperator{\im}{\text{Im}}
%\DeclareMathOperator{\PP}{\mathrm{P}}
\newcommand{\PP}[1]{\mathrm{P}^#1 \!}

%
% Make marginpars smaller font
%
\let\oldmarginpar\marginpar
\renewcommand\marginpar[1]{\oldmarginpar[\scriptsize #1]{\scriptsize #1}}

\newcommand{\thchar}[2] {\begin{bmatrix}#1\\#2\end{bmatrix}}
\newcommand{\thcharsm}[2] {\left[ \begin{smallmatrix} #1 \\ #2 \end{smallmatrix} \right]}

\title{Computing with Riemann Surfaces and Abelian Functions \\ {\small
    General Examination}}

\author{
Chris Swierczewski\\
University of Washington\\
Department of Applied Mathematics}
\date{\today}


%%%%%%%%%%%%%%%%%%%%%%%%%%%%%%%%%%%%%%%%%%%%%%%%%%%%%%%%%%%%%%%%%%%%%%%%%%%%%%%
\begin{document}
%%%%%%%%%%%%%%%%%%%%%%%%%%%%%%%%%%%%%%%%%%%%%%%%%%%%%%%%%%%%%%%%%%%%%%%%%%%%%%%

\maketitle

\begin{abstract}
The goal of my thesis is to develop the software tools necessary for
computing with complex algebraic curves, Riemann surfaces and Abelian
functions: periodic meromorphic functions $f : \CC^g \to \CC$ with $2g$
independent periods. Although Abelian functions first arose in the study
of Abelian integrals, they find application in solving non-linear
integrable partial differential equations, complex algebraic geometry,
and optimization. Algebraic curves and Riemann surfaces form the natural
environment for studying these functions. The purpose of this document
is to present the theory and algorithms going into these computations,
demonstate the implementation of these algorithms in the Python software
library {\tt abelfunctions (http://www.cswiercz.info/abelfunctions)},
and present my goals for this research.
\end{abstract}

%%%%%%%%%%%%%%%%%%%%%%%%%%%%%%%%%%%%%%%%%%%%%%%%%%%%%%%%%%%%%%%%%%%%%%%%%%%%%%%%
\section{Introduction}
%%%%%%%%%%%%%%%%%%%%%%%%%%%%%%%%%%%%%%%%%%%%%%%%%%%%%%%%%%%%%%%%%%%%%%%%%%%%%%%

The Kadomtsev-Petviashvili (KP) equation is a partial differential
equation used to describe the surface height of a two-dimensional
periodic shallow water wave. Depending on certain physical
considerations, which we will ignore, one can derive either of the
following two equations
\begin{align}
  \left(-4u_t + 6uu_x + u_{xxx}\right)_x + 3\sigma^2 u_{yy} = 0, \quad
  \sigma^2 = -1, \label{eqn: KP1} \\
  \left(-4u_t + 6uu_x + u_{xxx}\right)_x + 3\sigma^2 u_{yy} = 0, \quad
  \sigma^2 = +1, \label{eqn: KP2}
\end{align}
where $u(x,y,t)$ is the surface height as a function of position $(x,y)$
and time $t$. In the sequel we do not rely on this distinction and we
simply refer to the {\it KP equation}.

The KP equation admits a large family of quasiperiodic solutions of the
form
\begin{equation} \label{eqn: kpsol}
  u(x,y,t) = 2 \partial_x^2 \log \theta(Ux+Vy+Wt+z_0, \Omega) + c,
\end{equation}
where $\theta$ is the Riemann theta function.

\begin{definition} \label{def: riemanntheta}
  The {\bf Riemann theta function} $\theta: \CC^g \times \hh_g \to \CC$
  is defined in terms of its Fourier series:
  \begin{equation} \label{eqn: riemanntheta}
    \theta(z,\Omega) = \sum_{n \in \ZZ^g}
    e^{2 \pi i \left( \tfrac{1}{2} n \cdot \Omega n + n \cdot z \right)}.
  \end{equation}
  This function converges absolutely and uniformly on compact sets in
  $\CC^g \times \hh_g$ where $\hh_g$ is the space of all {\it ``Riemann
    matrices''} --- complex symmetric matrices with positive definite
  imaginary part.
\end{definition}

From the definition, we see that the Riemann theta function is periodic
in $z$ with integer periods and quasi-periodic in $z$ in the columns of
$\Omega$. In other words, if $m,n \in \ZZ^g$ then
\begin{equation} \label{eq: quasiperiodicity}
    \theta(z + m + \Omega n, \Omega) =
    e^{-2 \pi i \left( \tfrac{1}{2} n \cdot \Omega n + n \cdot z \right) }
    \theta(z, \Omega).
\end{equation}

A generalization of the Riemann theta function, involving a non-integer
shift in some of its arguments, is referred to as the Riemann theta
function with characteristics.

\begin{definition} \label{def: thetachar}
Let $\alpha,\beta \in [0,1)^{g}$. The {\bf Riemann theta function with
characteristic $\thcharsm{\alpha}{\beta}$} is defined as
\begin{align*}
  \theta\thchar{\alpha}{\beta}(z, \Omega) &=
  \sum_{n \in \mathbb{Z}^g}
  e^{2 \pi i \left( \tfrac{1}{2} (n+\alpha) \cdot \Omega (n+\alpha) +
    (n + \alpha) \cdot (z + \beta) \right) } \\
  &=
  e^{2\pi i \left( \tfrac{1}{2} \alpha \cdot \Omega \alpha +
    \alpha \cdot (z + \beta) \right)}
  \theta(z + \Omega \alpha + \beta, \Omega).
\end{align*}
\end{definition}

Note that $\theta \thcharsm{0}{0}(z,\Omega) = \theta(z,\Omega)$. See
\cite{NIST:DLMF,MumfordI07,MumfordII07} for further definitions and
properties of the Riemann theta function.

These solutions \eqref{eqn: kpsol} are the so-called {\it finite genus
  solutions} to the KP equation and families of such solutions exist for
every $g > 0$. In fact, the totality of solutions of this form are dense
the space of all periodic solutions to KP \cite{Krichever93}. The
constants $c\in\CC$, $U,V,W,z_0 \in \CC^g$ and $\Omega \in \hh_g$, as
well as the {\it genus} $g$, are determined from a Riemann surface. Any
such Riemann surface can produce a family of solutions to KP
\cite{Dubrovin81}. We postpone the definition of these constants in
terms of known quantities until more machinery is developed in the
following sections.

In general, periodic solutions to integrable partial differential
equations are {\it Abelian functions}.
\begin{definition} \label{def: abelian-function}
  An {\bf Abelian function} $f : \CC^g \to \CC$ of genus $g \geq 1$ is a
  meromorphic function such that there exists $2g$ vectors
  $w_1,\ldots,w_{2g} \in \CC^g$ linearly independent over the real
  numbers where
  \[
      f(z + w) = f(z),
  \]
  for all $z \in \CC^g$. When $g=1$ these are the {\it elliptic
    functions}.
\end{definition}
Abelian functions first arose from the study of Abelian integrals
\[
    \int_{z_0}^{z_1} \frac{P(z,w)}{Q(z,w)}dz,
\]
where $P,Q \in \CC[z,w]$ and $z$ and $w$ are related by an algebraic
equation $f(z,w) = 0$ with $f \in \CC[z,w]$.

Riemann theta functions play a central role in the theory of Abelian
functions in that all Abelian functions can be written as a rational
function of the Riemann theta function and its derivatives (such as in
the KP solution above).  The primary focus of my study is the
construction and numerical evaluation of these Abelian functions,
particularly those arising in applications.

Abelian functions are applicable in fields other than nonlinear water
waves. For example, they make explicit many computations such as those
in the study of solitary waves, black hole space-times, and algebraic
curves. One basic example is the calculation of bitangent lines of plane
algebraic curves; these are useful for computations in
optimization-related fields such as algebraic geometry and convex
optimization. Bitangents can used to represent smooth complex plane
quartic curves as either a symmetric determinant of a linear form or as
a sum of three squares \cite{PSV11}. In convex optimization, bitangents
are used to construct a {\it visibility complex} which, in turn, is used
to solve the shortest path problem in Euclidean space
\cite{PocchiolaVegter93}.

\begin{definition} \label{def: bitangent}
  A {\bf bitangent} to a plane algebraic curve $C : f(x,y) = 0, f \in
  \CC[x,y]$ is a line $\mathcal{L} \subset \CC$ that lies tangent to $C$
  at at least two distinct points.
\end{definition}

By Bezout's Theorem, if a curve has a bitangent it necessarily must be
of degree at least four \cite{Bezout1779}. A result of Pl\"{u}cker
determines that a degree four complex curve admits exactly 28 complex
bitangents \cite{Plucker34}. In particular, Pl\"{u}cker showed that the
number of real bitangents of any real quartic must be 28, 16, or fewer
than 9. The connection between Riemann theta functions and the bitangent
lines of smooth quartics was known to Riemann \cite{Baker97,Riemann76}
and, in fact, can be computed using the tools developed in this
research. See Figure \ref{fig: edge} for an example.

%% %!!!!!!!!!!!!!!!!
%% %!!!!!!!!!!!!!!!!
%% \marginpar{Note: Figure \ref{fig: edge} will be replaced with the one
%%   generated by abelfunctions once I have it working properly again /
%%   once I find the image on my computer.}
%% %!!!!!!!!!!!!!!!!
%% %!!!!!!!!!!!!!!!!

\begin{figure}[t]
  \centering
  \includegraphics[width=0.6\textwidth]{images/bitangents.jpg}
  \caption{The real graph of the Edge Quartic $C: f(x,y) = 25(x^4+y^4+1)
    - 34(x^2y^2+x^2+y^2) = 0$ (in blue) and its 28 real bitangents (in
    grey). Note that four of them lie tangent to $C$ at infinity. These
    lines were computed using the Riemann theta function.}
  \label{fig: edge}
\end{figure}

Finally, Riemann theta functions and algebraic curves can be used to
compute linear matrix representations of algebraic curves. A theorem
from classical algebraic geometry states that every homogenous
polynomial $f \in P^2\CC[x_0,x_1,x_2]$ can be written in the form
\[
   f(x_0,x_1,x_2) = \text{det}
   \left( A x_0 + B x_1 + C x_2 \right),
\]
where $A,B,C$ are symmetric complex matrices which can be efficiently
computed using Riemann theta functions. Furthermore, when the polynomial
has real coefficients then $A,B,C$ are symmetric real matrices and such
representations are important in the study of spectrahedra --- the
solution spaces of semidefinite programs \cite{PSV10}.

The purpose of my research is to develop efficient and performant
algorithms for computing with Abelian functions on Riemann surfaces. The
computational tools developed in this research program have far-reaching
and varied applications.


%%%%%%%%%%%%%%%%%%%%%%%%%%%%%%%%%%%%%%%%%%%%%%%%%%%%%%%%%%%%%%%%%%%%%%%%%%%%%%%%
\section{Complex Algebraic Geometry --- Period Matrices and the Abel Map}
%%%%%%%%%%%%%%%%%%%%%%%%%%%%%%%%%%%%%%%%%%%%%%%%%%%%%%%%%%%%%%%%%%%%%%%%%%%%%%%

This section serves as a brief introduction to the theory of complex
algebraic curves. Primary references are \cite{Ueno97,Griffiths89}.

%..............................................................................
\subsection{The Projective Line}
%..............................................................................

The primary motivation behind complex projective geometry is to make concrete
the way in which we analyze the behavior of functions, such as
polynomials, at infinity without having to resort to techniques separate
from those used at finite points. For example, in applications we may
need to integrate a differential along a path on an algebraic curve
going to infinity. Knowing the geometry of the curve at infinity makes
such an operation computationally feasible.

In fact, anyone with an elementary complex analysis background has seen
an example of projective geometry. The Riemann sphere is the complex
plane $\CC$ with a ``point at infinity'' added. Let $z$ denote the
coordinate in $\CC$ (i.e., the point $z=0$ represents the origin of the
complex plane). In order to discuss the point at infinity we introduce
the coordinate $w = 1/z$. The analysis of some function at $\infty$ is
equivalent to rewriting the problem in terms of the coordinate $w$ and
examining its behavior in a neighborhood of $w=0$. This explains why,
for example, the exponential function
\[
    e^z = \sum_{n=0}^\infty z^n / n!,
\]
though entire in the complex plane, has an essential singularity on the
Riemann sphere since the exponential function in the coordinate $w$
centered at $w=0$ is expressed by the series
\[
    \sum_{n=0}^\infty \frac{w^{-n}}{n!}.
\]

This point at infinity is not rigorously defined because it does not
make sense to {\it equate} $z=\infty$. The definition of the Riemann
sphere is made explicit by the following construction: consider the set
$U = \CC^2 - \{(0,0)\}$. Define the equivalence relation
\[
    (a_0, a_1) \sim (\lambda a_0, \lambda a_1),
    \quad \forall \lambda \in \CC - \{0\}.
\]
Thus two points $(a_0,a_1)$ and $(b_0,b_1)$ in $U$ are considered the
same if the ratios $a_0 : a_1$ and $b_0 : b_1$ are equal. The set of all
points $(b_0,b_1)$ equal to $(a_0,a_1)$ is called the {\it equivalence
  class} of $(a_0,a_1)$ and the {\it complex projective line} $\PP{1}\CC$
is the set of all such equivalence classes. That is,
\[
    \PP{1}\CC := \CC^2 / \sim.
\]
The equivalence class of $(a_0,a_1)$, called a ``point'' in $\PP{1}\CC$,
is written $(a_0 : a_1) \in \PP{1}\CC$. $\PP{1}\CC$ is precisely the
Riemann sphere. To see this, consider the two subsets
\begin{align*}
    U_0 &= \{ (a_0 : a_1) \in \PP{1}\CC \; | \; a_0 \neq 0 \}, \\
    U_1 &= \{ (a_0 : a_1) \in \PP{1}\CC \; | \; a_1 \neq 0 \}.
\end{align*}
For any $(a_0 : a_1) \in U_0$ we have, by the equivalence property,
\[
    (a_0 : a_1) = (1 : a_1/a_0) = (1 : a).
\]
Similarly, $(b_0 : b_1) = (b : 1)$ for every point in $U_1$. Every point
in the intersection $U_0 \cap U_1$ can be written in either of these two
forms. Each of these subspaces are isomorphic to $\CC$ since the maps
\begin{align*}
  \phi_0 : U_0 \to \CC,
  \quad
  \phi_0 \left( (a_0 : a_1) \right) = a_1 / a_0,
  & \quad \text{ and } \\
  \phi_1 : U_1 \to \CC,
  \quad
  \phi_1 \left( (a_0 : a_1) \right) = a_0 / a_1, &
\end{align*}
are continuous bijections with inverses
\begin{gather}
  \phi^{-1}_0(a) = (1 : a), \\
  \phi^{-1}_1(b) = (b : 1).
\end{gather}
Finally, note that $(0 : 1)$ is the only projective point in $U_1$ which
is not in $U_0$. Therefore, we identify $U_0$ with the complex plane (in
the coordinate $z$) and the point $P_\infty = (0 : 1)$ with the point at
infinity and set
\begin{equation} \label{eqn: projective-line}
  \PP{1}\CC = U_0 \cup \{ (0 : 1) \} \cong \CC \cup P_\infty.
\end{equation}

Indeed $P_\infty$ is considered the point at infinity on the Riemann
sphere for if one considers the image of $(0 : 1)$ under $\phi_0$,
though undefined since $(0:1) \not \in U_0$, it maps to $z = 1 / 0$
``='' $\infty$. Again, this does not make sense without the complex
projective space construction above but is merely used to illustrate the
point. The coordinate transformation from $z$ to $w$ at the beginning of
this section is equivalent to identifying $U_1$ with the complex plane
$\CC$ and $\{(1 : 0)\}$ with the point at infinity, instead.

%------------------------------------------------------------------------------
\subsection{The Projective Plane}
%------------------------------------------------------------------------------

The natural environment we use in the sequel is not the complex
projective line but the complex projective plane. In this section we
construct the projective plane and examine its geometric properties. The
construction is similar to that of the projective line.

Let $U = \CC^3 - \{(0,0,0)\}$. Following the strategy of the previous
section, consider the set of all ratios $(a_0 : a_1 : a_2)$, that is,
the collection of all equivalence classes under the equivalence relation
$(a_0 : a_1 : a_2) \sim (\lambda a_0 : \lambda a_1 : \lambda a_2),
\forall \lambda \in \CC - \{0\}$. The space of all such equivalence
classes is called the two-dimensional complex projective space or {\it
  the projective plane} and is denoted $\PP{2}\CC$.

Define the subsets $U_0,U_1,U_2$ by
\[
    U_j = \{ (a_0 : a_1 : a_2) \in \PP{2}\CC \; | \; a_j \neq 0 \},
\]
and note that all $(a_0 : a_1 : a_2) \in U_0$ satisfy $(a_0 : a_1 : a_2)
= (1 : a_1/a_0 : a_2/a_0)$. We define the bijective mapping
\begin{gather*}
    \phi_0 : U_0 \to \CC^2, \\
    \phi_0 \left( (a_0 : a_1 : a_2) \right)
    =
    \left( \frac{a_1}{a_0}, \frac{a_2}{a_0} \right), \\
    \phi_0^{-1} \left( (x,y) \right)
    =
    (1 : x : y).
\end{gather*}
The mappings $\phi_1$ and $\phi_2$ are similarly defined on $U_1$ and
$U_2$, respectively. Therefore, we can identify $U_0$ with the complex
plane $\CC^2$.

Consider the space $U_0^c = \PP{2}\CC - U_0$. By definition, every point
in $U_0^c$ is of the form $(0 : a_1 : a_2)$. By definition, every point
in $U_0^c$ determines a point on the complex projective line
$\PP{1}\CC$. The converse is true as well, resulting in the bijection
\[
    (0 : a_1 : a_2) \in \PP{2}\CC
    \; \leftrightarrow \;
    (a_1 : a_2) \in \PP{1}\CC.
\]
By identifying $U_0^c$ with $\PP{1}\CC$ we may write
\begin{equation} \label{eqn: proejctive-plane}
  \PP{2}\CC = U_0 \cup U_0^c \cong \CC^2 \cup \PP{1}\CC
\end{equation}
where $U_0^c \cong \PP{1}\CC$ is called the {\it line at infinity},
denoted $l_\infty$, and $U_0 \cong \CC^2$ is called the {\it complex
  affine plane}. We may also identify the complex affine plane with the
sets $U_1$ or $U_2$ and the line at infinity with their complements.

We saw a natural geometric interpretation of $\PP{1}\CC$ in the previous
section. Does such an interpretation exist for $\PP{2}\CC$? Consider a line
in the complex affine plane $\CC^2$ which can be written in the form
\[
    \alpha + \beta x + \gamma y = 0,
    \quad \text{where }
    (\beta,\gamma) \neq 0, \alpha, \beta, \gamma \in \CC.
\]
Using the inverse mapping $\phi_0^{-1}$ on $\CC^2$ we have
\[
    x = \frac{x_1}{x_0} \text{ and } y = \frac{x_2}{x_0},
\]
where $(x_0 : x_1 : x_2)$ are the coordinates of $\PP{2}\CC$, and we get
the line
\[
    \alpha x_0 + \beta x_1 + \gamma x_2 = 0.
\]
This equation, called the {\it homogenization} of the affine curve,
makes sense in all of $\PP{2}\CC$. Setting $x_0 = 1$ gives the original
affine line. On the other hand, setting $x_0 = 0$ gives the equation
\[
    \beta x_1 + \gamma x_2 = 0,
\]
which is the equation of the line in $l_\infty$. However, this implies
$x_1 / x_2 = - \gamma / \beta$. Hence the projective point $(0 : -\gamma
: \beta)$ satisfies the equation
\[
    \alpha x_0 + \beta x_1 + \gamma x_2 = 0
\]
and is, in fact, the only projective point in $l_\infty$ on the line.

This means that the line ``intersects'' $l_\infty$ at the point $(0 :
-\gamma : \beta)$ and that this intersection point depends only on the
slope of the affine portion of the line. Hence, the line at infinity has
the geometric meaning that each point on it is the intersection point of
an entire family of parallel lines in $\CC^2$. This leads to a
generalization of a theorem from classical planar geometry: {\it any
  two, distinct lines in $\PP{2}\CC$ intersect at exactly one point}.

%------------------------------------------------------------------------------
\subsection{Projective Plane Curves}
%------------------------------------------------------------------------------

The set of all points $(x_0, x_1, x_2)$ satisfying
\[
    \alpha x_0 + \beta x_1 + \gamma x_2 = 0
\]
is called a projective line and is a simple example of a projective
algebraic curve (of degree one). In this section we introduce various
properties of general projective curves.

An {\it complex plane algebraic curve} is the zero locus of the
homogenization of a polynomial $f \in \CC[x,y]$. That is, given a
polynomial $f(x,y) = \alpha_n(x) y^n + \alpha_{n-1}(x)y^{n-1} + \cdots +
\alpha_0(x)$ its homogenization is the polynomial $F \in
\PP{2}\CC[x_0,x_1,x_2]$ where
\[
    F(x_0,x_1,x_2) = x_0^d f(x_1/x_0,x_2/x_0).
\]
where $d$ is the degree of $F$. The homogeneity of $F$ means that we can
write
\[
    F(x_0,x_1,x_2) = \sum_{i+j+k=d} \alpha_{ijk} x_0^i x_1^j x_2^k.
\]

In terms of the projective polynomial $F$, its affine part can be
written $f(x,y) = F(1,x,y)$. As in the case of a projective line, $f$
can be thought of as a projection of the polynomial $F$ onto $\CC^2$ and
there is always a one-to-one correspondence between an affine polynomial
and its homogenization. Therefore, a {\it complex plane algebraic curve}
is the set
\[
  C = \left\{
  (x_0 : x_1 : x_2) \in \PP{2}\CC : F(x_0,x_1,x_2) = 0
  \right\}.
\]

Important to the study of projective curves, and specifically in the
computational work described here, are singular points. A point $a =
(a_0 : a_1 : a_2) \in C$ is a {\it singular point of $C$}, or a {\it
  multiple point of $C$}, if
\[
  \left(
    \frac{\partial F}{\partial x_0},
    \frac{\partial F}{\partial x_1},
    \frac{\partial F}{\partial x_2}
  \right) (a)
  = (0,0,0).
\]
Consider the case when $a = (1 : 0 : 0)$ (corresponding to the point
$(0,0)$ in the affine plane $\CC^2$) is a singular point of $F$. The
affine poirtion of the curve is
\[
f(x,y) = \sum_{i+j \geq 2}^d c_{ij} x^iy^j.
\]
Note that the constant term is zero since $(0, 0)$ is a point on the
affine curve and the linear term vanishes since $(0,0)$ is a singular
point. We write
\[
  f(x,y) = f_m(x,y) + f_{m+1}(x,y) + \cdots + f_d(x,y), \quad m \geq 2,
\]
where each $f_n$ is the sum of all terms of $f$ of degree $n$; that is,
terms of the form $c_{ij}x^iy^j$ such that $i+j=n$. The smallest such
$m$ with non-zero term $f_m$ appearing in $f$ is called the {\it
  multiplicity} of the singular point $(1 : 0 : 0)$. Singularities with
multiplicity two are called {\it double points}, those with multiplicity
three are called {\it triple points}, and so on.

The homogeneous term $f_m$ can be factored into linear factors
\[
  f_m(x,y) = \prod_{j=1}^m (\alpha_j x - \beta_j y).
\]
We call the space $f_m(x,y) = 0$ the {\it tangent cone of the plane
  curve $C$} at $a = (1 : 0 : 0)$ consistsing of a finite number of
intersecting lines $L_j : \alpha_j x - \beta_j y$.

When a generic affine point $a = (1 : c : d)$ is a singular point we
write the affine curve in the form
\[
    f(x,y) = \sum_{i+j \geq 2}^d \tilde{c}_{ij} (x-c)^i(y-d)^j
\]
which we can write as a sum of polynomials $g_n(x-c,y-d)$ homogenous in
$x-c$ and $y-d$.

In the case when the singular point $a = (0 : 1 : b) \in l_\infty$ we
repeat the above process with the affine curve
\[
    g(u,v) = \frac{1}{x_1^d} F(x_0, x_1, x_2) = F(u,1,v),
    \quad u = \frac{x_0}{x_1}, v = \frac{x_2}{x_1},
\]
which is a projection of $F$ onto $U_1 \cong \CC$ instead of $U_0$.  We
write $g$ as a sum of terms of the form $g_{ij}u^i(v-b)^j$. Finally, in
the case $a = (0 : 0 : 1) \in l_\infty$ we use the affine curve
\[
    h(w,z) = \frac{1}{x_2^d} F(x_0, x_1, x_2) = F(w,z,1),
    \quad w = \frac{x_0}{x_2}, z = \frac{x_1}{x_2},
\]
and write $h$ as a sum of terms of the form $h_{ij}w^iz^j$.

\begin{example} \label{ex: cubic}
Consider the cubic curve
\[
    C: F(x_0,x_1,x_2) =
    x_0^4 x_2^3 + 2 x_0^3 x_1^3 x_2 - x_1^7
\]
In complex affine space $x_0 = 1$ this curve is
\[
    f(x,y) = F(1,x,y) = y^3 + 2 x^3 y - x^7.
\]
A plot of $f$ for $x,y$ real is shown in Figure \ref{fig: example-cubic}.
For $a = (a_0 : a_1 : a_2)$ we have
\begin{align} \label{eq: singular-conditions}
    \frac{\partial F}{\partial x_0}(a)
    &=
    4 a_{0}^{3} a_{2}^{3} + 6 a_{0}^{2} a_{1}^{3} a_{2}, \notag \\
    \frac{\partial F}{\partial x_1}(a)
    &=
    6 a_{0}^{3} a_{1}^{2} a_{2} - 7 a_{1}^{6}, \notag \\
    \frac{\partial F}{\partial x_2}(a)
    &=
    3 a_{0}^{4} a_{2}^{2} + 2 a_{0}^{3} a_{1}^{3}.
\end{align}

\begin{figure}
  \centering
  \includegraphics[width=0.8\textwidth]{images/2_example_curve.png}
  \caption{A real plot of the curve $f(x,y) = y^3 + 2 x^3 y - x^7$.}
  \label{fig: example-cubic}
\end{figure}

First, we find the finite singular points of $C$. Setting $a_0=1$ and
solving the above equations for $a_1$ and $a_2$ we see that $p = (1 : 0
: 0)$ is the only finite singular point of $F$. Note that
\begin{gather*}
    f(x,y) = f_3(x,y) + f_4(x,y) + f_7(x,y) \\
    f_3(x,y) = y^3, \; \;
    f_4(x,y) = 2 x^3 y, \; \;
    f_7(x,y) = -x^7,
\end{gather*}
and that $f_3$, $f_4$, and $f_7$ are homogeneous of degrees 3, 4, and 7,
respectively. Therefore, $p$ is a singular point of multiplicity 3 with
\[
    f_3(x,y) = y^3 = 0,
\]
as the equation for the tangent cone at $p$. These properties are
suggested by Figure \ref{fig: example-cubic} where, near the point $p$,
the real curve looks like the intersection of three curves well
approximated by the line $y = 0$ at $x = 0$.

Setting $a_0 = 0$, the only expression in Equation \eqref{eq:
  singular-conditions} that does not reduce to zero is
\[
    \frac{\partial F}{\partial x_1}((0,a_1,a_2))
    =
    - 7 a_{1}^{6} = 0,
\]
implying that the point $a = (0 : 0 : 1)$ is the only singular point at
infinity. The curve at infinity centered at $(0 : 0 : 1)$ is
\[
    h(w,z) = F(w,z,1) = w^{4} + 2 w^{3} z^{3} - z^{7}.
\]
The order of this singularity is four since this is the degree of the
lowest degree homogeneous term. The tangent cone at $a$ is $g_4(w,z) =
w^4$.
\end{example}


%------------------------------------------------------------------------------
\subsection{Connection to Riemann Surfaces}
%------------------------------------------------------------------------------

There is a close relationship between the study of compact Riemann
surfaces and that of algebraic curves. Recall that a Riemann surface
$\tilde{C}$ is a complex manifold of complex dimension one endowed with
an {\it atlas}: an open covering $\{U_\alpha \}_{\alpha \in A}$ of
$\tilde{C}$ together with a collection of homeomorphisms $\{z_\alpha :
U_\alpha \to \CC\}_{\alpha \in A}$, called {\it local parameters}, such
that every pair of {\it transition functions}
\[
    f_{\beta,\alpha} := z_\beta \circ z_\alpha^{-1} :
    z_\alpha \left( U_\alpha \cap U_\beta \right)
    \to
    z_\beta \left( U_\alpha \cap U_\beta \right),
\]
is holomorphic. The pairs $(U_\alpha, z_\alpha)$ are called {\it
  coordinate charts}. In other words, a Riemann surface is a topological
space such that for all points $P \in \tilde{C}$ there is a neighborhood
of $P$ homeomorphic to an open subset of the complex plane and one can
analytically continue along the surface via transition functions.

The Riemann sphere $\tilde{C} = \CC^*$ is an example of a Riemann
surface. Its atlas consists of two coordinate charts $(U_1, z_1)$ and
$(U_2, z_2)$ with
\begin{align*}
    U_1 = \CC, & \quad z_1 = z, \\
    U_2 = \left( \CC - \{0\} \right) \cup \{ \infty \}, & \quad z_2 = 1/z.
\end{align*}
This is a valid atlas since the transition functions
\begin{gather*}
    f_{1,2}, f_{2,1} : \CC - \{0\} \to \CC - \{0\} \\
    f_{1,2} = z_1 \circ z_2^{-1} = 1/z \\
    f_{2,1} = z_2 \circ z_1^{-1} = 1/z
\end{gather*}
are holomorphic on $U_1 \cap U_2 = \CC - \{0\}$.

This relationship between curves and Riemann surfaces are embodied by
the following two theorems \cite{Griffiths89}.

\begin{theorem} \label{thm: normalization}
  {\it (Normalization Theorem.)} For any irreducible algebraic curve $C
  \subset \PP{2}\CC$ there exists a compact Riemann surface $\tilde{C}$ and
  a holomorphic mapping
  \[
      \sigma : \tilde{C} \to \PP{2}\CC,
  \]
  such that $\sigma( \tilde{C} ) = C$ and $\sigma$ is injective on the
  inverse image of the set of smooth points of $C$.
\end{theorem}

A Riemann surface together with the mapping $\sigma$ is called the {\it
  normalization of $C$}. Loosely speaking, the normalization theorem
states that an algebraic curve is a Riemann surface except at the
singular points.

Conversely, every compact Riemann surface can be represented by an
algebraic curve.

\begin{theorem} \label{thm: repr-theorem}
  Any compact Riemann surface $\tilde{C}$ can be obtained through the
  normalization of a certain plane algebraic curve $C$ with at most
  ordinary double points. That is, there exists a holomorphic mapping
  \[
      \sigma : \tilde{C} \to \PP{2}\CC
  \]
  such that $\sigma(\tilde{C})$ is an algebraic curve possessing at most
  ordinary double points.
\end{theorem}

In the algorithms presented in this document we primarily work with the
affine part $f(x,y)$ of the curve $F(x_0,y_0,z_0)$. If analysis on the
line at infinity is necessary, when computing singular points for
example, we consider an affine projection $g$ of $F$ onto $l_\infty$,
\[
    g(u,y) = u^d f(1/u,y) = 0.
\]
Thus, the Riemann surface considered here is the branched algebraic
$y$-covering of the complex $x$-Riemann sphere. That is, the set of all
$(x,y)$-solutions $\tilde{C}$ to the affine polynomial equation
\[
    f(x,y) = \alpha_d(x)y^d + \alpha_{d-1}y^{d-1} + \cdots +
             \alpha_1(x)y + \alpha_0(x) = 0
\]
as $x$ varies along all of $\PP{1}\CC$. We treat $x$ and $y$ as the
independent and dependent variables of the equation, respectively.

A {\it point} $\alpha \in \CC$ is called a {\it regular point of $f$} if
\[
    f(\alpha,y) = 0
\]
has $d$ distinct $y$-roots $y_0,\ldots,y_{d-1}$. A point $\alpha \in
\CC$ is called a discriminant point if it in not regular. $\alpha =
\infty$ is a regular point of $f$ if
\[
    g(0,y)
\]
has $d$ disctinct roots.

A {\it place} $P$ is an element of $\tilde{C}$. For all but finitely
many places, $P$ is given by a pair $(\alpha,\beta)$ such that
$f(\alpha,\beta) = 0$. However, some places, particularly those where
$\alpha$ is a discriminant point, need instead be represented by a pair
of series $(x(t),y(t))$ in some local coordinate $t$. This will be
discussed in more detail in the following chapter.

%%%%%%%%%%%%%%%%%%%%%%%%%%%%%%%%%%%%%%%%%%%%%%%%%%%%%%%%%%%%%%%%%%%%%%%%%%%%%%%%
\section{Computing Period Matrices of Algebraic Curves and the Abel Map}
%%%%%%%%%%%%%%%%%%%%%%%%%%%%%%%%%%%%%%%%%%%%%%%%%%%%%%%%%%%%%%%%%%%%%%%%%%%%%%%

In this section we will look at algorithmically computing period
matrices. Each subsection here examines in close detail a major
component of the algorithm and provides a theoretical overview of the
component, an algorithm for computing the component, and an example
presented in {\tt abelfunctions}, a Python software package

%------------------------------------------------------------------------------
\subsection{abelfunctions}
%------------------------------------------------------------------------------

%------------------------------------------------------------------------------
\subsection{Puiseux Series}
%------------------------------------------------------------------------------

%..............................................................................
\subsubsection*{Theory}
%..............................................................................

%..............................................................................
\subsubsection*{Algorithm}
%..............................................................................

\begin{algorithm}[h]
\caption{POLYGON --- Returns the Newton polygon of the polynomial $F =
  F(X,Y)$.}
\label{alg: puiseux-polygon}
\begin{algorithmic}[1]
\Input

$F,X,Y$ --- A polynomial.

$I$ --- If $I=0$, return the Newton polygon. If $I=1$, return only the
segments of the Newton polygon with negative slope. $I=1$ is used to
compute terms beyond the first of the Puiseux series.

\Output A set of lists $(q,m,l,\Phi)$ where
\begin{itemize}
  \item $q,m,l$ defines a line segment $\Delta: qj+mi=l$ in the $(i,j)$
    plane,
  \item $\Phi(Z) = \sum_{(i,j)\in\Delta} a_{ij} Z^{(i-i_0)/q} \in
    \CC[Z]$ where $i_0$ is the smallest value of $i$ such that there is
    a point $(i_0,j)\in\Delta$.
\end{itemize}

\Function{POLYGON}{$F,X,Y,I$}
  \State
\EndFunction
\end{algorithmic}
\end{algorithm}


\begin{algorithm}[h]
\caption{REGULAR --- Given the branching or singular part of a Puiseux
  series, computes the regular part of the series.}
\label{alg: puiseux-regular}
\begin{algorithmic}[1]
\Input

$S$ --- a finite set of pairs $\{(\pi_k,F_k)\}$

$X,Y$ --- the dependent and independent variables, respectively

$H$ --- bound on the number of desired terms of the series

\Output $R$ --- a finite set of pairs $\{(\pi_k,F_k)\}$ containing at
least $H$ terms

\Function{Regular}{$S,X,Y,H$}
  \State $R \leftarrow ()$
  \ForEach{$(\pi,F)$ {\bf in} $S$}
    \While{$\text{len}(\pi) < H$}

      \State $m \leftarrow \text{min} \{ j \; /$
      \Call{COEFFICIENT}{$F,0,j$} $ \; | \; j \neq 0 \}$

      \State $\beta \leftarrow$ \Call{COEFFICIENT}{$F,0,m$} /
      \Call{COEFFICIENT}{$F,1,0$}

      \State $\tau \leftarrow (1,1,m,\beta)$
      \State $\pi \leftarrow \pi \cup \{\tau\}$
      \State $F \leftarrow$ \Call{NEWPOLYNOMIAL}{$F,\tau,m$}
    \EndWhile
    \State $R \leftarrow R \cup \{\pi\}$
  \EndFor
\EndFunction
\end{algorithmic}
\end{algorithm}

%..............................................................................
\subsubsection*{Examples}
%..............................................................................

\begin{lstlisting}
from abelfunctions import *
from sympy.abc import x,y,t

alpha = 0
f = y**8 + x*y**5 + x**4 - x**6
C = RiemannSurface(f,x,y)
p = C.puiseux(alpha, nterms=3, parametric=t)

for pi in p:
    sympy.pprint(pi)
\end{lstlisting}
\begin{pyoutput}
          11    7      
  5    6*t     t     3 
(t , - ----- - -- - t )
         25    5       
           9    5     
   3    2*t    t      
(-t , - ---- - -- + t)
         3     3      
\end{pyoutput}

%------------------------------------------------------------------------------
\subsection{Singularities}
%------------------------------------------------------------------------------

%
\subsubsection*{Theory}
%
%
\subsubsection*{Algorithm}
%
%
\subsubsection*{Examples}
%

%------------------------------------------------------------------------------
\subsection{Holomorphic Differentials}
%------------------------------------------------------------------------------

%
\subsubsection*{Theory}
%



%
\subsubsection*{Algorithm}
%
%
\subsubsection*{Examples}
%

%------------------------------------------------------------------------------
\subsection{Analytic Continuation}
%------------------------------------------------------------------------------

%
\subsubsection*{Theory}
%

A path on a Riemann surface is a continuous map $\gamma : [0,1] \to C
\subset \CC^2$. That is, if $\gamma(t) = (x_\gamma(t), y_\gamma(t))$
then $f(x(t),y(t)) = 0$ for all $t \in [0,1]$.


%
\subsubsection*{Algorithm}
%

Since analytic continuation is repeatedly performed when computing
period matrices it is important to make it performant.

Often, a path in $\CC_x$, $x_\gamma$, will be specified, such as via the
{\sc monodromy() or homology()} functions, and the corresponding
$y$-values lying above need to be determined. A naive approach to doing
so is to use a numerical root finder.

An alternate approach, which is the one used here, is to use Newton's
method --- given $x_\gamma(t_k)$ and some $y_{\gamma,j}(t_k)$ lying
above determine $y_{\gamma,j}(t_{k+1})$ above $x_\gamma(t_{k+1})$.

%
\subsubsection*{Examples}
%

%------------------------------------------------------------------------------
\subsection{Monodromy}
%------------------------------------------------------------------------------

%
\subsubsection*{Theory}
%
%
\subsubsection*{Algorithm}
%
%
\subsubsection*{Examples}
%

%------------------------------------------------------------------------------
\subsection{Homology}
%------------------------------------------------------------------------------

%
\subsubsection*{Theory}
%
%
\subsubsection*{Algorithm}
%
%
\subsubsection*{Examples}
%

%------------------------------------------------------------------------------
\subsection{Period Matrices}
%------------------------------------------------------------------------------

%
\subsubsection*{Theory}
%
%
\subsubsection*{Algorithm}
%
%
\subsubsection*{Examples}
%

%------------------------------------------------------------------------------
\subsection{The Abel Map}
%------------------------------------------------------------------------------

%
\subsubsection*{Theory}
%
%
\subsubsection*{Algorithm}
%
%
\subsubsection*{Examples}
%

%%%%%%%%%%%%%%%%%%%%%%%%%%%%%%%%%%%%%%%%%%%%%%%%%%%%%%%%%%%%%%%%%%%%%%%%%%%%%%%
\section{Future Work}
%%%%%%%%%%%%%%%%%%%%%%%%%%%%%%%%%%%%%%%%%%%%%%%%%%%%%%%%%%%%%%%%%%%%%%%%%%%%%%%

{\tt abelfunctions} provides a collection of tools for computing on
Riemann surfaces. Although a few more features need to be added most of
the package's functionality is ready to be used to solve problems. In
this chapter, I will present several problems that I wish to address for
my thesis:
\begin{itemize}
  \item Solving the I.V.P. for periodic solutions to non-linear,
    integrable, partial differential equations.
  \item Provide a framework for constructing and computing rational
    functions on Riemann surfaces with prescribed poles and zeros.
  \item Efficiently computing linear matrix representations of
    homogenous curves.
\end{itemize}



%------------------------------------------------------------------------------
\subsection{Periodic Solutions to Integrable Partial Differential Equations}
%------------------------------------------------------------------------------



We return to the Kadomtsev--Petvishvili (KP) equation given in the
introductory chapter of this document
\begin{equation*}
    \left(-4u_t + 6uu_x + u_{xxx}\right)_x + 3\sigma^2 u_{yy} = 0.
\end{equation*}
As mentioned, the KP equation has a large class of periodic solutions of
the form
\begin{equation}\label{eqn: solnformula}
    u(x,y,t) = 2 \partial_x^2 \log \theta(Ux + Vy + Wt + z_0, \Omega) + c
\end{equation}
called the ``theta function solutions''. The Schottky problem states
that every complex plane curve $C$, and therefore every period matrix
$\tau = [I \; \Omega]$, produces a Riemann matrix which appears is the
solution form above.

Given a divisor $D = \sum_i n_i P_i$ on $C$ the parameters $U,V,W,z_0
\in \CC^g$ and $c \in \CC$ can be determined by integrating certain
meromorphic differentials around certain paths on $C$. The algorithms
and infrastucture needed to define divisors and compute these quantities
will be the first problem I address in my thesis work.

The use of a divisor as input to this solution generator is rather
abstract. Deconinck \cite{Deconinck98} provides a method for determining
a divisor from a set of initial data to the KP equation, thus alowing a
more ``physical'' input to the solution algorithm. That is, the
machinery for computing solutions of the form above can be used to solve
the initial value problem to KP. The process of determining this initial
data from wave measurements is a difficult problem and will not be
addressed in this work. My goal will be to provide a means of
computationally solving the initial value problem for $KP$.

One may ask, ``Where does the solution formula in Equation \ref{eqn:
  solnformula} come from?'' An extension to this program will be to
provide an algorithm for determining the form of the theta function
solutions to arbitrary integrable equations and subsequently computing
these solutions. The form of the theta function solution to any given
integrable PDE can be determined from the Lax pair formulation of the
problem
\[
    u_t = F(u,u_x,\ldots)
    \quad \leadsto \quad
    \frac{dX}{dt} - \frac{dT}{dx} = \left[ T,X \right]
\]
where $X$ and $T$ are square matrix operators depending on
$u,u_x,\ldots$ and a complex parameter $\lambda$. The relevant plane
curve can be determined from the Lax pair formulation of a PDE. I would
like to provide a black-box algorithm for computing the theta function
solutions to this class of PDEs.


%------------------------------------------------------------------------------
\subsection{The Schottky--Klein Prime Form}
%------------------------------------------------------------------------------


In addition to providing the means of computing paths and 1-forms on a
Riemann surface $C$ it would be nice to have a way of constructing
functions, other than the Abel map, which are defined on $C$. For
example, is it possible to construct and subsequently compute
meromorphic functions with prescribed zeros and poles on $C$? That is,
does a function $E : C \times C \to \CC$ exist such that $E(P,Q) = 0$ if
and only if $P = Q$? It turns out that such a function ``almost'' exists
yet satisfied enough properties to make the function useful.

\begin{definition}
  A {\bf non-singular odd theta characteristic} $[\delta] = [\alpha,
    \beta]$ is a theta characteristic where, for a given $\alpha,\beta
  \in \{0,1/2\}^g$,
  \begin{itemize}
    \item $\nabla\theta[\alpha,\beta](0,\Omega) \neq \mathbf{0}$ and
    \item $4 \alpha \cdot \beta \equiv 1 \pmod{2}$.
  \end{itemize}
  A {\bf non-singular even theta characteristic} is a theta
  characteristic where, instead,
  \[
      4 \alpha \cdot \beta \equiv 0 \pmod{2}.
  \]
\end{definition}

\begin{definition}
  The {\bf Schottky--Klein prime form} $E:C \times C \to \CC$ is defined
  by
  \[
    E(P,Q) =
    \frac{
      \theta[\delta]
      \left( \int_{P}^{Q}\mathbf{\omega},\Omega \right)
    }
    {
      \sqrt{\zeta(P)}\sqrt{\zeta(Q)}
    }
    =
    \frac{
      \theta[\delta]
      \left( A(Q) - A(P), \Omega \right)
    }
    {
      \sqrt{\zeta(P)}\sqrt{\zeta(Q)}
    }
  \]
  where $\omega = (\omega_j)_{j=1}^g$ is the vector of the normalized
  basis of holomorphic 1-forms of $C$, $A:C \to \CC^g$ is the
  Abel--Jacobi map, and for a given non-singular odd theta
  characteristic $[\delta]$
  \[
    \zeta
    = \nabla \theta[\delta](0,\Omega) \cdot \omega
    = \sum_{j=1}^g \frac{\partial}{\partial z_j}
    \theta[\delta](0,\Omega) \omega_j.
  \]
\end{definition}

Unfortunately, $E$ is not holomorphic on $C \times C$ nor is it even
well defined in part because it depends on the choice of path from $P_1$
to $P_2$. However, it is holomorphic and well-defined on $\tilde{C}
\times \tilde{C}$ where $\tilde{C}$ is the universal cover of the curve
$C$ which, in this case, is the cut surface $\hat{C}$ [XXX]. The good
news is that the zeros of $E$ are independent of the choice of
representative from the universal cover: if $(\tilde{P}_1,\tilde{Q}_1)
\in \tilde{C} \times \tilde{C}$ and $(\tilde{P}_2,\tilde{Q}_2) \in
\tilde{C} \times \tilde{C}$ have the same projection $(P,Q) \in C \times
C$ then $E(\tilde{P}_1,\tilde{Q}_1) = 0$ if and only if
$E(\tilde{P}_2,\tilde{Q}_2) = 0$.

As a result, one can use the Schottky--Klein prime form to define
meromorphic funtions on a Riemann surface $C$. Let
$P_1,\ldots,P_m,Q_1,\ldots,Q_n \in C$. Then the function
\[
    f : C \to \CC, \quad
    f(P) = \frac{\prod_{i=1}^m E(P,P_i)}{\prod_{j=1}^n E(P,Q_j)}
\]
has zeros at the places $P_1,\ldots,P_m$ and poles at the places
$Q_1,\ldots,Q_n$. The ability to efficiently construct and quickly
evaluate the prime form would make rational functions on Riemann
surfaces as computationally accesible as Abelian functions.

%% \begin{enumerate}
%%   \item $E(P,Q) = 0$ if and only if $\tilde{P} = \tilde{Q}$.
%%   \item $E(P,Q) = - E(Q,P)$.
%%   \item $E$ is invariant under $a$-periods: if
%%     \[
%%     P \mapsto P' = P + \sum m_i A_i
%% \end{enumerate}

%------------------------------------------------------------------------------
\subsection{Linear Matrix Representations}
%------------------------------------------------------------------------------

The Schottky--Klein prime form, as fundamental and interesting as it is
in its own right, appears in construction of linear matrix
representations of certain plane curves. A theorem from classical
algebraic geometry states that every complex homogenous polynomial in
three variables can be written as
\[
    F(x,y,z) = \det \left( Ax + By + Cz \right)
\]
where $A,B,C$ are symmetric matrices. Such a representation is called a
{\it linear matrix representation}. Linear matrix representations of
polynomial appear in problems in control theory and can be used to solve
polynomial inequalities via semidefinite programming.

The curves we consider here come from spectrahedra. A {\it
  two-dimensional spectrahedron} is a subset of $\RR^2$ bounded by
rigidly convex algebraic curves; real curves with a maximal number of
nested ovals in the real plane. The interior of the innermost oval of
such a curve defines a spectrahedron. For example, the real projective
curve
\[
    F(x_0,x_1,x_2) =
    2x_0^4 + x_1^4 + x_2^4 - 3x_0^2x_1^2 - 3x_0^2x_2^2 + x_1^2x_2^2.
\]
of degree four with affine part
\[
    f(x,y) = x^{4} + x^{2} y^{2} - 3 x^{2} + y^{4} - 3 y^{2} + 2
\]
has $4/2 = 2$ nested ovals, as shown in Figure \ref{fig:
  spectrahedron}. These curves are called {\it Helton--Vinnikov curves}
and, by the Helton-Vinnikov theorem, completely characterize all
two-dimensional spectrahedra \cite{HeltonVinnikov07}.

\begin{figure}[t]
  \centering
  \includegraphics[width=0.7\textwidth]{images/helton-vinnikov.png}
  \caption{A real plot of the Helton-Vinnikov curve $f(x,y) = x^{4} +
    x^{2} y^{2} - 3 x^{2} + y^{4} - 3 y^{2} + 2$. The region bounded by
    the innermost oval is a spectrahedron.}
  \label{fig: spectrahedron}
\end{figure}

In some applications, it is important to have the matrices $A,B,C$ be
real when the curve $F$ is real. The representations important to
studying spectrahedra also require that the linear matrix represntation
is a {\it real definite representation}; that is, that the span of
$A,B,$ and $C$ contain a real positive definite matrix. Plaumann,
Sturmfels, and Vinzant \cite{PSV10} study several approaches to
computing real definite matrices of Helton--Vinnikov curves.

One approach, in fact the one chosen by Helton and Vinnikov, gives a
positive definite linear matrix representation of a Helton-Vinnikov
curve in terms of theta functions and the Schottky--Klein prime form.

\begin{theorem}
{\bf (Helton--Vinnikov)} Let $C : f(x,y,z) = 0$ be a real homogenous
curve of degree $d$ with $f(1,0,0) = 0$ and assume
\begin{enumerate}
  \item $C$ is a Helton--Vinnikov curve with the point $f(1,0,0) = 0$
    inside the innermost oval,
  \item the $d$ real intersection points with the line $\{z = 0\}$ are
    distinct non-singular points $Q_1,\ldots,Q_d$ with coordinates $Q_i
    = (-\beta_i : 1 : 0), \beta_i \neq 0$,
  \item $\delta$ is an even theta characteristic with
    $\theta[\delta](0,\Omega) \neq 0$.
\end{enumerate}
Then,
\[
    f(x,y,z) = \det \left( I_{d \times d}x + By + Cz \right)
\]
where $B = \text{diag}(\beta_1,\ldots,\beta_d)$, $C$ is a real
symmetric matrix with diagonal entries
\[
    c_{ii} = \beta_i
    \frac{\partial_z f(-\beta_i,1,0)}{\partial_y f(-\beta_i,1,0)},
\]
and off-diagonal entries
\[
    c_{jk} = \frac{\beta_k - \beta_j}{\theta[\delta](0,\Omega)}
    \frac{\theta[\delta](A(Q_k) - A(Q_j))}{E(Q_j,Q_k)},
\]
where $E : C \times C \to \CC^g$ is the Schottky--Klein prime form.
\end{theorem}

The calculation of linear matrix representations of Helton--Vinnikov
curves is an excellent application of the Schottky--Klein prime form and
I would like to provide an algorithm for doing so.


%%%%%%%%%%%%%%%%%%%%%%%%%%%%%%%%%%%%%%%%%%%%%%%%%%%%%%%%%%%%%%%%%%%%%%%%%%%%%%%
\section{Bibliography}
%%%%%%%%%%%%%%%%%%%%%%%%%%%%%%%%%%%%%%%%%%%%%%%%%%%%%%%%%%%%%%%%%%%%%%%%%%%%%%%


\bibliographystyle{amsplain}
\bibliography{general-exam}


%%%%%%%%%%%%%%%%%%%%%%%%%%%%%%%%%%%%%%%%%%%%%%%%%%%%%%%%%%%%%%%%%%%%%%%%%%%%%%%
\end{document}
%%%%%%%%%%%%%%%%%%%%%%%%%%%%%%%%%%%%%%%%%%%%%%%%%%%%%%%%%%%%%%%%%%%%%%%%%%%%%%%
