%%%%%%%%%%%%%%%%%%%%%%%%%%%%%%%%%%%%%%%%%%%%%%%%%%%%%%%%%%%%%%%%%%%%%%%%%%%%%%%
\section{Computing Period Matrices}
%%%%%%%%%%%%%%%%%%%%%%%%%%%%%%%%%%%%%%%%%%%%%%%%%%%%%%%%%%%%%%%%%%%%%%%%%%%%%%%

This section introduces the concepts and algorithms needed to compute
period matrices of Riemann surfaces. Each subsection examines a major
concept behind this calculation and provides a theoretical overview of
the component, a brief description of the algorithm used to compute the
component, and examples presented in Python using the software library
{\tt abelfunctions}.

Please note that {\tt abelfunctions}, though largely functional, is
still in early stages of development. Therefore, the library syntax
presented in this document may not match the syntax of future
versions. Consult the documentation located at
\verb=www.cswiercz.info/abelfunctions= for up to date information on the
package.

The two primary ingredients involved in computing period matrices are a
{\it basis of closed cycles} on a Riemann surface $X$ and a {\it basis
  of holomorphic differentials} on $X$. Both of these depend on the
components discussed in this section. The dependency relationship of all
of these components is outlined in Figure \ref{fig: dependencies}, which
forms the outline of this section. The book chapter
\cite{DeconinckPatterson11} serves as a primary reference for this
section.

\begin{figure}
\centering
\begin{tikzpicture}
  \tikzstyle{element}=[rectangle, rounded corners, thick, draw]
  \tikzstyle{leadsto}=[->, shorten >=5pt, shorten <=5pt, >=latex, thick]

  % period matrix
  \node[element] (periodmatrix) {Period Matrix};

  % dummy element to the left of period matrix
  \node[left=of periodmatrix, xshift=-1cm] (dummy) {};

  % = algebraic side =
  \node[element, above=of dummy] (oneforms) {1-Forms}
      edge [leadsto] (periodmatrix);
  \node[left=of oneforms, xshift=-0.5cm] (algdummy) {}; % alg dummy
  \node[element, above=of algdummy] (intbasis) {Integral Basis}
      edge [leadsto] (oneforms);
  \node[element, below=of algdummy] (singularities) {Singularities}
      edge [leadsto] (oneforms);
  \node[element, left=of algdummy,xshift=-0.2cm] (puiseux) {Puiseux Series}
      edge [leadsto] (singularities)
      edge [leadsto] (intbasis);

  %% \draw [->] (puiseux.east) -- (intbasis.west);
  %% \draw [->] (puiseux.east) -- (singularities.west);
  %% \draw [->] (intbasis.east) -- (oneforms.west);
  %% \draw [->] (singularities.east) -- (oneforms.west);
  %% \draw [->] (oneforms.east) -- (periodmatrix.west);

  % = geometric side =
  \node[element, below=of dummy] (homology) {Homology}
      edge [leadsto] (periodmatrix);
  \node[element, left=of homology] (monodromy) {Monodromy}
      edge [leadsto] (homology);
  \node[element, left=of monodromy, text width=2.1cm, align=center] (ancont) {Analytic Continuation}
      edge [leadsto] (monodromy);

  %% \draw [->] (ancont.east) -- (monodromy.west);
  %% \draw [->] (monodromy.east) -- (homology.west);
  %% \draw [->] (homology.east) -- (periodmatrix.west);
\end{tikzpicture}
\caption{The major computations performed by {\tt abelfunctions} and
  their dependencies on one another.}
\label{fig: dependencies}
\end{figure}


%------------------------------------------------------------------------------
\subsection{Puiseux Series}
%------------------------------------------------------------------------------

%..............................................................................
\subsubsection*{Theory}
%..............................................................................

Every analytic function $f = f(x)$ admits a local Taylor series
representation in a neighborhood about $x = \alpha$. If the function is
meromorphic it still admits a local series representation in the form of
a Laurent series. An extension of Taylor series are the Laurent series
\[
    f(x) = \sum_{n=N}^\infty c_n (x-\alpha)^n
\]
for some $N \in \ZZ \cup \{-\infty\}$ depending on $\alpha$. In both of
these situations the variable $x$ is a {\it local coordinate} of $f$
near the point $\alpha$.

For algebraic curves, local coordinates are given in terms of Puiseux
series, which can be thought of as an extension of Laurent series.
\begin{definition} \label{def: puiseux}
  A {\bf Puiseux series} expansion of a curve $C : f(x,y) = 0$ at the
  point $x=\alpha$ is a collection of $j = 1,\ldots,m \leq d =
  \text{deg}_y f$ series of the form
  \begin{align*}
      P_j(t) =
      \begin{dcases}
        x_j(t) = \alpha + \lambda_j t^{e_j}, \\
        y_j(t) = \sum_{k=N}^\infty \beta_{jk} t^{n_{jk}},
      \end{dcases}
  \end{align*}
  where $N \in \ZZ \cup \{-\infty\}$, $\alpha, \lambda_j, \beta_{jk} \in
  \CC$, and $e_j,n_{jk} \in \ZZ$. Each Puiseux series $P_j$ is a
  ``place'' on $C$.
\end{definition}
A place $P_j = P_j(t)$ satisfies
\[
    f\big(P_j(t)\big) := f\big(x_j(t), y_j(t)\big) = 0.
\]
When a Puiseux series $P_j(t)$ represents an expansion about a
non-singular $(\alpha, \beta_j)$ on the curve then $P(0) =
(\alpha,\beta)$. This is not necessarily the case about singular
places. We list some additional important facts and properties of
Puiseux series.
\begin{itemize}
  \item The integer $|e_j|$ is called the {\it branching number} or {\it
    ramification index} of the series expansion at that place: $|e_j| >
    1$ when $x = \alpha$ is a branch point of the curve.
  \item The number of Puiseux series $m$ at a branch point $x = \alpha$
    is strictly less than $d = \text{deg}_y f$. However, $d =
    \sum_{j=1}^m |e_j|$.
  \item The field of Puiseux series is a splitting field for $\CC[x,y] =
    \CC[x][y]$. That is, given any $f \in \CC[x,y]$ and Puiseux series
    expansions about any $x=\alpha$ we can write
    \[
        f(x,y) = \prod_{j=1}^m \prod_{k=1}^{e_j} \left(
                 y - y_j\left(
                     (e/\lambda_j)^{2 \pi ik / e_j}(x-\alpha)
                     \right)
                 \right),
    \]
    where the first product ranges over all Puiseux series $P_j$ at
    $x=\alpha$ and the second product ranges over all $y$-components
    $y_j(x_j)$ when solving for $t$ in terms of $x$. Note that a Puiseux
    series $P_j$ with ramification index $|e_j|>1$ splits into $|e_j|$
    $y$-series in $x$.
\end{itemize}

%..............................................................................
\subsubsection*{Algorithm}
%..............................................................................

The algorithm used in {\it abelfunctions} for computing truncations of
Puiseux series expansions is based on that of Duval \cite{Duval89}. The
main ingredient of the algorithm are Newton polygons of algebraic
curves. We give a brief outline of the process here.
\begin{itemize}
  \item The goal of the algorithm is to compute a list of tuples $\pi =
    (\tau_1, \tau_2, \ldots, \tau_R)$ where $\tau_i =
    (q_i,\mu_i,m_i,\beta_i,\eta_i)$. These tuples define the relations
    \begin{align*}
      X_{i-1} &= \mu_i X_i^{q_i}, \\
      Y_{i-1} &= (\beta_i + \eta_iY_i)X_i^{m_i},
    \end{align*}
    where $i = 1, \ldots, R$. To obtain the desired Puiseux series we
    set $x = X_0$, $y = Y_0$, and $t = X_R$ and eliminate the
    intermediate variables $X_1,\ldots,X_{R-1}$ and
    $Y_1,\ldots,Y_{R-1}$.
  \item The first set of $\tau_i$ computed are those representing the
    {\it singular part} of $P$, that is, the part of the series if the
    Puiseux series has a ramification index $|e_j|>1$. This is done
    using the Newton polygon method. The output of this stage of the
    algorithm provides enough information to distinguish the Puiseux
    series expansions at $x=\alpha$.
  \item Finally the algorithm computes the {\it regular} terms of the
    Puiseux series using a standard Taylor series techniques.
\end{itemize}

%..............................................................................
\subsubsection*{Examples}
%..............................................................................

\begin{example} \label{ex: 3-puiseux}
  Consider the curve
  \[
      C : f(x,y) = y^3 + 2x^3y - x^7 = 0.
  \]
  As seen in Example \ref{ex: 2-cubic}, the point $(x,y) = (0,0)$ is a
  singular point of $C$. The Puiseux series expansions lying above $x=0$
  are all of the form
  \begin{align*}
    P_1(t) &=
    \begin{dcases}
      x(t) = t, \\
      y(t) = \frac{t^{4}}{2} - \frac{t^{9}}{16} + \frac{3 t^{14}}{128} + \cdots,
    \end{dcases} \\
    P_2(t) &=
    \begin{dcases}
      x(t) = - \frac{t^{2}}{2}, \\
      y(t) =  - \frac{t^{3}}{2} - \frac{t^{8}}{64} + \frac{3 t^{13}}{4096} + \cdots.
    \end{dcases}
  \end{align*}
  We compute these expansions using {\tt abelfunctions}.

  \begin{lstlisting}
  from abelfunctions import *
  from sympy.abc import x,y,t

  f = y**3 + 2*x**3*y - x**7
  alpha = 0

  P = puiseux(f, x, y, alpha, nterms=3, parametric=t)

  print 'Puiseux Expansions at x = \%s:'\%(alpha)
  for Pj in P:
      sympy.pprint(Pj)
      print
  \end{lstlisting}
  \begin{pyoutput}
  Puiseux Expansions at x = 0:
         14    9    4 
      3*t     t    t  
  (t, ----- - -- + --)
       128    16   2  

     2      13    8    3 
   -t    3*t     t    t  
  (----, ----- - -- - --)
    2     4096   64   2  
  \end{pyoutput}
\end{example}

\vspace{24pt}

MORE EXAMPLES WILL BE INSERTED HERE.

\vspace{24pt}


%------------------------------------------------------------------------------
\subsection{Singularities}
%------------------------------------------------------------------------------

%
\subsubsection*{Theory}
%

Recall from Definition \ref{def: singular-point} that a point $a$ on a
projective curve $C$ is a singular point if
\[
    \left(
      \frac{\partial F}{\partial x_0},
      \frac{\partial F}{\partial x_1},
      \frac{\partial F}{\partial x_2}
    \right) (a)
    = (0,0,0).
\]
For singular points of the form $a = (1 : \alpha, \beta)$, this is
equivalent to
\[
    \frac{\partial f}{\partial x} (\alpha,\beta) = 0, \quad
    \frac{\partial f}{\partial y} (\alpha,\beta) = 0,
\]
where $f$ is the affine portion of the curve. The singular points of a
curve need to be determined not only for the numerical analytic
continuation and integration methods discussed below, so we can
appropriately desingularize the curve $C$ and obtain a Riemann surface
$X$, but they are also an essential ingredient to computing a basis of
holomorphic 1-forms on $X$.

A major role of Puiseux series is to provide a local coordinate chart at
a singular point. For singular points of the form $a = (1 : \alpha :
\beta)$ the Puiseux series expansion $P_j$ of $f = f(x,y)$ such that
$P_j(0) = (\alpha, \beta)$ is a coordinate chart centered at $(x,y) =
(\alpha, \beta)$. $P_j$ tells us how to approach and pass through $(x,y)
= (\alpha, \beta)$ on the curve. With this coordinate chart and those at
other singular points of $C$ we can desingularize the curve and thus
create an appropriate atlas for the corresponding Riemann surface $X$.

For the purposes of computing the genus of $X$ as well as the space of
holomorphic 1-forms $\Gamma(X,\Omega_X^1)$ on $X$ we need to compute the
delta invariant and the multiplicity of a singularity, respectively. The
following discussion assumes the singularity is finite. To analyze
infinite singular points we project the curve $C$ onto the line at
infinity $l_\infty$ using the method described in Section \ref{sec:
  projective-plane-curves}.

{\bf Branching number.} The branching number $R$ of a singular point
$(\alpha,\beta)$ is the sum of the branch numbers of the Puiseux series
expansions centered at $(x,y) = (\alpha,\beta)$. That is,
\[
    R \quad = \sum_{\substack{j \\ P_j(0)=(\alpha,\beta)}} |e_j|.
\]

{\bf Multiplicity.} As given in Section \ref{sec:
  projective-plane-curves}, the multiplicity of a singular point is the
degree of the lowest degree non-zero homogeneous term appearing in the
polynomial expression for the curve centered at $(\alpha, \beta)$.

{\bf Delta invariant.} The delta invariant $\delta_P$ of a singularity
$P$ is the number of double points concentrated at the singularity. This
is equal to the number of quadratic factors $(\alpha_i x - \beta_i y)^2$
appearing in the tangent cone at the singularity. Let $S$ be the set of
all singular points, finite and infinity, of $C$. Then the genus is
given by
\begin{equation} \label{eqn: genus-formula}
    g = \frac{(d-1)(d-2)}{2} - \sum_{P \in S} \delta_P.
\end{equation}


%
\subsubsection*{Algorithm}
%

First we determine the finite singularities of $C : f(x,y) = 0$. Let
$R(x)$ be the resultant of $f(x,y)$ and $\partial_y f(x,y)$
\cite{Griffiths89}. We compute the roots
\begin{align*}
    S &= \{ x \in \CC \; | \; R(x) = 0, \partial_x R(x) = 0 \} \\
      &= \{ x_1, \ldots, x_s \}.
\end{align*}
For each $x_j \in S$ we compute the $y$-roots $\{y_{j1},\ldots,y_{jd}\}$
where $d = \text{deg}_y f$. The places $(x_j,y_{jk})$, $j=1,\ldots,s$,
$k=1,\ldots,d$ satisfy
\[
    f(x_j,y_{jk}) = 0, \quad \partial_y f(x_j,y_{jk}) = 0,
\]
by the definition of the resolvent. Therefore, the singular places
$(x_k,y_{jk})$ are those that satisfy
\[
    \partial_x f(x_j,y_{jk}) = 0.
\]
By definition, these remaining places are the finite singular points of
the curve $C$. For the infinite case we use the projection of the curve
on the line at infinity. (See Section \ref{sec:
  projective-plane-curves}.)

%% To compute the branching number of a singular point $(\alpha,\beta)$ we
%% simply calculate the sum of the ramification indices at all Puiseux
%% series $P_j$ such that $P_j(0) = (\alpha,\beta)$.

%% We can also use Puiseux series to compute the multiplicity of a singular
%% point.

The methods used to compute the branching number, multiplicity, and
delta invariant of a singularity rely on examining the leading order
behavior of the Puiseux series expansions such that $P_j(0) = (\alpha,
\beta)$. For the details of these methods see
\cite{DeconinckPatterson11}. In brief the branching number $R$ of
singularity is computed using the above formula
\[
    R \quad = \sum_{\substack{j \\ P_j(0)=(\alpha,\beta)}} |e_j|.
\]
The multiplicity is the sum of the minimum of $e_j$ and $n_{jN}$ over
all Puiseux series $P_j$ such that $P_j(0) = (\alpha,\beta)$ where
$\beta_{jN} t ^{n_{jN}}$ is the first non-zero, non-constant term
appearing the in $y_j$. The delta invariant is equal to
\[
    \delta = \frac{1}{2}\sum_{j=1}^m r_j \text{Int}_{P_j} - r_j + 1,
\]
where
\[
    \text{Int}_{P_j} = \sum_{k=1^d, k \neq j}
    \text{val}_x \big( y_j(x) - \tilde{y}_k(x) \big),
\]
with $\text{val}_x( g(x))$ equal to the lowest exponent of $x$ appearing
in $g(x)$, and $y_j(x)$ is the $y$-part of $P_j$ when solving for $t =
t(x)$. The sum appearing in $\text{Int}_{P_j}$ is taken over {\it all}
Puiseux series expansions $P_k$ at $x = \alpha$, not just the ones with
$P_k(0) = (\alpha, \beta)$.



%
\subsubsection*{Examples}
%

We compute the finite singularities $a = (1 : \alpha : \beta)$ and the
infinite singularities $a = (0 : 1 : \gamma)$ of the curve
\[
    C : f(x,y) = y^3 + 2x^3y - x^7 = 0.
\]
\begin{lstlisting}
from abelfunctions import *
from sympy.abc import x,y

f = y**3 + 2*x**3*y - x**7
S = singularities(f,x,y)

for s,(m,delta,r) in S:
    if s[0] == 0:
        print 'Infinite Singularity at:', s
    else
        print 'Finite Singularity at:  ', s
    print '  multiplicity     =', m
    print '  delta invariant  =', delta
    print '  branching number =', r
    print
\end{lstlisting}
\begin{pyoutput}
Finite Singularity at:   (1, 0, 0)
  multiplicity     = 3
  delta invariant  = 4
  branching number = 2

Infinite Singularity at: (0, 1, 0)
  multiplicity     = 4
  delta invariant  = 9
  branching number = 1
\end{pyoutput}
The genus is computed using the {\tt genus()} function which, in
addition to using the genus formula in Equation \eqref{eq:
  genus-formula}, performs additional checks on the genus using
algebraic and geometric properties discussed below.
\begin{lstlisting}[firstnumber=16]
d = 7  # the homogenous degree of f is 7
g = (d-1)*(d-2)/2

for s,(m,delta,r) in S:
    g -= delta

print 'degree       =', d
print 'genus        =', g
print 'genus(f,x,y) =', singularities.genus(f,x,y)
\end{lstlisting}
\begin{pyoutput}
degree       = 7
genus        = 2
genus(f,x,y) = 2
\end{pyoutput}

%------------------------------------------------------------------------------
\subsection{Holomorphic 1-Forms}
%------------------------------------------------------------------------------

%
\subsubsection*{Theory}
%

1-forms on a Riemann surface $X$ are objects that can be integrated
along piecewise smooth paths on $X$.
\begin{definition}
  {\bf (1-Form)} Let $X$ be a Riemann surface with atlas $\{ (U_\alpha,
  z_\alpha) \}$. A 1-form $\omega$ on $X$, also called a differential,
  is such that in each local coordinate $z_\alpha : U_\alpha \subset X
  \to \CC$,
  \[
      \omega \Big|_{U_\alpha} = f_\alpha(z_\alpha) dz_\alpha,
  \]
  and the appropriate compatibility conditions are satisfied under the
  action of transition functions on $U_\alpha \cup U_\beta$ where
  $(U_\beta, z_\beta)$ is another local coordinate. The space of all
  1-forms on $X$ is denoted $\Omega_X^1$.
\end{definition}
The space of all {\it holomorphic 1-forms} is of particular interest.
\begin{definition}
  {\bf (Holomorphic 1-Forms)} The space of holomorphic 1-forms
  $\Gamma(X,\Omega_X^1)$ on $X$ is the space of 1-forms $\omega$ such
  that in each local coordinate $(U_\alpha, z_\alpha)$,
  \[
      \omega \Big|_{U_\alpha} = h_\alpha(z_\alpha) dz_\alpha
  \]
  where $h_\alpha : U_\alpha \to \CC$ is a holomorphic function.
\end{definition}
For a compact genus $g$ Riemann surface $X$, $\Gamma(X,\Omega_X^1)$ is a
finite-dimensional vector space of dimension $g$ over $\CC$. Thus, it
has a basis of $g$ holomorphic 1-forms $\{\omega_1, \ldots, \omega_g\}$.

For Riemann surfaces obtained by desingularizing and compactifying an
algebraic curve $C : f(x,y) = 0$ these basis holomorphic 1-forms can be
written as
\begin{equation*}
  \omega_k(x,y) = \frac{p_k(x,y)}{\partial_y f(x,y)} dx,
\end{equation*}
where $p_k \in \CC[x,y]$ is of degree at most $d-3$ in $x$ and $y$. The
polynomials $p_k$ are called the {\it adjoint polynomials of $f$}. Note
that since $y$ has explicit dependence on $x$ due to the equation
$f(x,y) = 0$, we can use $x$ as the local coordinate of the
differential.

%
\subsubsection*{Algorithm}
%

One condition on the $p_k$'s is immediately apparent: to preserve
holomorphicity $p_k$ must have a zero, with sufficient multiplicity, at
the places $P = (\alpha,\beta)$ where $\partial_y f(x,y)$ vanishes. More
precisely, Noether showed that if a singular place $P$ has multiplicity
$m_P$ then $P_k$ must have a zero of order at least $m_P - 1$ at $P$
\cite{Noether83}.

The technique we use to determine the adjoint polynomials $p_k$ uses a
theorem of M\~{n}uk relying on computing an integral basis for the
algebraic function field of the curve \cite{Mnuk97}. Let $A(C)$ be the
{\it coordinate ring}
\[
    A(C) = \CC[x,y] / (f)
\]
of the curve $C : f(x,y) = 0$. The coordinate ring can be though of as
the ring of all functions $g \in \CC[x,y]$ vanishing on the curve
$f$. Note that $A(C)$ is a subset of the {\it algebraic function field}
$\CC(x,y)$.

Given a ring $R$ and a field $S$ such that $R \subset S$, the {\it
  integral closure $\bar{R}$ of $R$ in $S$} is the ring of all elements
$s \in S$ such that
\[
    s^n + r_{n-1}s^{n-1} + \cdots + r_1 s + r_0 = 0,
\]
for some choice of $n >0$ and $r_0,\ldots,r_{n-1} \in R$. That is
$\bar{R}$ consists of all elements in $S$ satisfying some monic
polynomial equation with coefficients in $R$. Here, we consider the
integral closure $\overline{A(C)}$ of $A(C)$ in $\CC(x,y)$.

Again, we wish to find the set of all adjoint polynomials of $C$. By
\cite{Mnuk97}, these are the set of all $p \in \CC[x,y]$ such that
\[
    \overline{A(C)} p(x,y) \subset\CC[x,y].
\]
That is, all of the polynomials $p$ such that every element of
$\overline{A(C)}$, when multiplied by $p$ and reduced modulo $f(x,y)$,
results in a polynomial. Now, $\overline{A(C)}$ is a finite extension of
$A(C)$. Therefore,
\[
    \overline{A(C)} = \beta_1 A(C) + \cdots + \beta_m A(C),
    \quad
    \beta_k \in \CC(x,y).
\]
for an appropriate choice of $\beta_k$'s in $\CC(x,y)$. The set
$\{\beta_1,\ldots,\beta_m\}$ is called the {\it integral basis} of the
integral closure of the coordinate ring. Thus, finding the set of
adjoint polynomials is equivalent to finding polynomials $p \in
\CC[x,y]$ such that
\[
    \beta_k(x,y) p(x,y) \in \CC[x,y], \quad \forall j=1,\ldots,m.
\]

To compute the adjoint polynomials we write
\[
    p(x,y) = \sum_{i+j \leq d-3} c_{ij}x^iy^j
\]
where $d = \text{deg}_y f$. We compute an integral basis
$\{\beta_1,\ldots,\beta_m\}$ for $\overline{A(C)}$ using the algorithm of
van Hoeij \cite{vanHoeij94}. The requirement that
\[
    \beta_k(x,y) \sum_{i+j \leq d-3} c_{ij}x^iy^j \in \CC[x,y]
\]
imposes a number of conditions on the coefficients $c_{ij}$ appearing in
the expression for $p(x,y)$. The set of all possible $c_{ij}$'s
satisfying these conditions for every $\beta_k, k=1,\ldots,m$ gives us
the adjoint polynomials we need.

%
\subsubsection*{Examples}
%

We compute a basis of holomorphic 1-forms on the Riemann surface $X$
given by the desingularization and compactification of the algebraic
curve
\[
    C : f(x,y) = y^3 + 2x^3y - x^7 = 0.
\]
\begin{lstlisting}
from sympy.abc import x,y,t

f = y**3 + 2*x**3*y - x**7
X = RiemannSurface(f,x,y)
oneforms = X.holomorphic_differentials()

for omega in oneforms:
    print 'omega(x,y) =\n'
    sympy.pprint(omega, use_unicode=False)
    print
\end{lstlisting}
\begin{pyoutput}
omega(x,y) =

    x*y    
-----------
   3      2
2*x  + 3*y 

omega(x,y) =

      3    
     x     
-----------
   3      2
2*x  + 3*y
\end{pyoutput}
From this we can infer that the adjoint polynomials of the curve are
$p_1(x,y) = xy$ and $p_2(x,y) = x^3$.


%------------------------------------------------------------------------------
\subsection{Analytic Continuation} \label{sec: analytic-continuation}
%------------------------------------------------------------------------------

%
\subsubsection*{Theory}
%

A {\it path on a Riemann surface} is a continuous map $\gamma : [0,1]
\to C \subset \CC^2$. That is, if $\gamma(t) = (x_\gamma(t),
y_\gamma(t))$ then $f(x(t),y(t)) = 0$ for all $t \in [0,1]$. The roots
of a polynomial are continuous as a function of the
coefficients. Therefore, an $x$-path $x_\gamma : [0,1] \to \CC_x$ and an
initial $y$-root $y_0 \in \CC_y$ are sufficient for defining a path on
$C$ for the resulting $y$-path $y_\gamma : [0,1] \to \CC_y$ is
completely determined by the curve
\[
    f(x_\gamma(t),y) = 0.
\]
The process of deriving this $y$-path from the data provided is
referred to as {\it analytic continuation}.

A {\it closed path $\gamma$ on a Riemann surface} is one such that
$\gamma(0) = \gamma(1)$. That is, a path is closed when $x_\gamma(0) =
x_\gamma(1)$ and $y_\gamma(0) = y_\gamma(1)$. When constructing a path
using an $x$-path it may be the case that the $x$-path $x_\gamma(t)$ is
closed in $\CC_x$ but the derived $y$-path $y_\gamma(t)$ may not satisfy
$y_\gamma(0) = y_\gamma(1)$. This situation is described in more detail
in Section \ref{sec: monodromy} on monodromy groups of algebraic curves.

%
\subsubsection*{Algorithm}
%

To compute a path $\gamma$ on a Riemann surface $C$ we provide a
continuous $x$-path $x_\gamma(t)$ and an initial $y$-value $y_0$ as
input and we wish to receive the resulting $y$-path $y_\gamma(t)$ as
output. Analytic continuation of $y_0$ along $\gamma$ is a fundamental
operation in {\tt abelfunctions} since evaluation of and integration
along paths is done frequently. Therefore, it is important to make the
construction and evaluation along paths as fast and efficient as
possible.

We use numerical methods to estimate values along $y_\gamma(t)$. In
general, the problem is phrased as given $\gamma(t_i) = (x_i,y_i)$ as
well as some later $t_{i+1} = t_i + \Delta t$ and $x_{i+1} =
x_\gamma(t_{i+1})$ determine the value $y_{i+1} = y_\gamma(t_{i+1})$.

A first and natural approach to solving this problem is to use a root
finder. Given $x_{i+1}$ we numerically or symbolically solve the
equation
\[
    f(x_{i+1},y) = 0.
\]
This produces $n$ $y$-roots $y_{i+1,1}, \ldots, y_{i+1,d}$ over
$x_{i+1}$. However, even if one finds an effective and fast method for
doing this with arbitrary degree polynomials $f$, the main problem with
this approach is determining which $y_{i+1,k}$ is equal to the desired
root $y_{i+1}$. One could argue that the desired root is the one
minimizing $|y_{i+1,k} - y_i|$ (the root closest to the previous
$y$-root) but it is conceivable that this closest can change as a
function of $\Delta t$, especially if $\Delta t$ is too large.

Another approach could be to use Newton iteration: given $x_{i+1}$ and
an {\it initial guess} $y_i$ at $x_i$ use Newton iteration on the
function $g(y) = f(x_{i+1}, y)$ to determine $y_{i+1}$. However, this
approach suffers from the same problem, namely, the root $y_{i+1,k}$
produced by Newton iteration may change as a function of $\Delta t$. Too
large of a $\Delta t$ may result in {\it branch jumping}, where we
converge to the incorrect $y$-root. Too small of a $\Delta t$ gives an
inefficient numerical algorithm. Further, the definition of ``small
enough'' may change as a function of the curve $C$ and the $x$-points
$x_{i}$ and $x_{i+1}$.

To solve the problem of selecting an appropriate $\Delta t$ we use
Smale's $\alpha$-theory \cite{Smale85}. The purpose of Smale's
$\alpha$-theory is to answer the following questions about $g(y) =
f(x_{i+1},y)$ for some finite set of points $Y \subset \CC$:
\begin{enumerate}
  \item From which points in $Y$ will Newton's method converge
    quadratically to some solution to $g$?
  \item From which points in $Y$ will Newton's method converge
    quadratically to distinct solutions to $g$?
  \item If $g$ is real ($\bar{g} = g$), from which points of $Y$ will
    Newton's method converge quadratically to real solutions to $g$?
\end{enumerate}
See \cite{HauensteinSottile10} for an excellent summary of Smale's
$\alpha$-theory. Using the notation of Hauenstein and Sottile, we
outline the analytic continuation algorithm here.
\begin{enumerate}
  \item Assume we know the {\it $y$-fibre} $y_i =
    \{y_{i,1},\ldots,y_{i,d}\}$ of $g_i(y) := f(x_i,y)=0$. Fix some
    initial $\Delta t$ and let $x_{i+1} = x_\gamma(t_i + \Delta t)$. We
    wish to compute the $y$-fibre $y_{i+1}$ of $g_{i+1}(y) :=
    f(x_{i+1},y)$ such that each element $y_{i+1,j}$ is the analytic
    continuation of $y_{i,j}$ to $x_{i+1}$.
  \item We first determine if the $y$-fibre $y_i$ is an {\it approximate
    solution} to $g_{i+1}(y) = 0$. Does each element of $y_i$ lie in the
    quadratic convergence region of Newton's method on $g_{i+1}$?

    If not, return to Step (1) with $\Delta t \mapsto \Delta t / 2$.
  \item Next, determine if the approximate solutions $y_{i,j}$ will
    converge to distinct associated solutions $y_{i+1,j}$: will the
    approximate solutions jump branches or stay on their respective
    branches?

    If not, return to Step (1) with $\Delta t \mapsto \Delta t / 2$.
  \item The $y$-fibre satisfies the necessary conditions for Newton
    iteration to converge to the appropriate analytic continuations
    $y_{i+1,j}$ at $x_{i+1}$. Newton iterate and output this solution
    $y$-fibre.
\end{enumerate}

Note that this algorithm requires analytically continuing all of the
$y$-roots along an $x$-path in the complex plane since we cannot
determine an appropriate step size for continuing a given root without
knowing the locations of the other roots. Although this impacts the
performance of the algorithm since we have to perform $d$ sets of Newton
iterations at each step, Smale's $\alpha$-theory provides a rigorous
method for determining an appropriate step size.


%------------------------------------------------------------------------------
\subsection{Monodromy} \label{sec: monodromy}
%------------------------------------------------------------------------------

%
\subsubsection*{Theory}
%

At a generic point $x = \alpha_0 \in \CC$ a curve $C : f(x,y) = 0$ has
$d$ distinct ordered $y$-roots $(y_0,\ldots,y_{d-1})$ at
$\alpha_0$. This collection of $y$-roots is sometimes called the {\it
  lift of} or the {\it fibre above} $x=\alpha_0$. However, at a point
$x=b$ where both $f(x,y) = 0$ and $\partial_y f(x,y) = 0$ the number of
distinct roots in the lift is strictly less than $d$. Such a point $x =
b$ is called a {\it discriminant point} of $f$.

A {\it branch point} $x=b$ is a discriminant point having the property
that if one were to analytically continue an ordered fibre around some
closed path encircling $b$ then the elements of the fibre are
permuted. Specifically, let $x_{\gamma} : [0,1] \to \CC$ be a piecewise
differentiable oriented closed path in the complex $x$-plane encircling
a branch point $x=b$ exactly once in the positive direction and let
$(y_0,\ldots,y_{d-1})$ be a fixed ordering of the fibre at $x_\gamma(0)
= b$. Then, after analytically continuing the fibre around $x_\gamma$
and returning to $x_\gamma(1) = b$, the fibre is equal to
\[
    (y_{\pi_b(0)}, \ldots, y_{\pi_b(d-1)}),
\]
where $\pi_b \in S_d$ is a permutation on $d$ elements. In other wods, a
{\it branch point} is a discriminant point with $\pi_b \neq \text{id}$.

To analyze the permutation behavior of multiple branch points
$\{b_1,\ldots,b_n\}$ we start by fixing some {\it base point}
$x=\alpha_0$ in the complex plane such that $\alpha_0$ is not a branch
point and we fix an ordering $(y_0,\ldots,y_{d-1})$ of the fibre above
$\alpha_0$. Let $x_{\gamma_k} : [0,1] \to \CC$ be a path encircling only
the branch point $b_k$ in the positive direction which does not cross
the other paths. Such a path is called a {\it monodromy path} of
$b_k$. In the case when $x = \infty$ is a branch point a monodromy path
for $\infty$ is taken to be a circle going around all of the finite
branch points in the negative direction. See Figure \ref{fig: mon} for
an illustration of these paths.

\begin{figure}
  \centering
%  \includegraphics[width=0.6\textwidth]{images/mon.png}

  \begin{tikzpicture}
    \tikzstyle{discpt}=[circle,draw=white,fill=black,thin,minimum size=2pt,
                        radius=2pt]
    \tikzstyle{monpath}=[decoration={markings,
        mark=at position 0.5 with {\arrow[very thick]{latex}}},
      postaction={decorate}]

    % draw the discriminant points and interpolating dots
    \node[discpt] (b1) at (-1,-2)     [label=right:$b_1$] {};
    \node[discpt] (b2) at (0,-1)      [label=right:$b_2$] {};
    \node at (0,0) {$\bullet$};
    \node at (-0.1,0.4) {$\bullet$};
    \node at (-0.3,0.8) {$\bullet$};
    \node[discpt] (bn) at (-1,2)     [label=right:$b_n$] {};
    \node[discpt] (a) at (-6,0)      [label=left:$\alpha_0$] {};

    % the oriented paths
    \draw[monpath] (-6,0) ..
                   controls (-0.5,-3.5) and (0,-3) ..
                   (0,-2);
    \draw[monpath] (0,-2) ..
                   controls (0,-1) and (-2.5,-1.2) ..
                   (-6,0);
    \node at (-1.5,-2.8) {$x_{\gamma_1}$};

    \draw[monpath] (-6,0) ..
                   controls (-2.5,1.2) and (0,1) ..
                   (0,2);
    \draw[monpath] (0,2) ..
                   controls (0,3) and (-0.5,3.5) ..
                   (-6,0);
    \node at (-1.5,0.5) {$x_{\gamma_n}$};

    % path around infinity
    \draw[monpath] (-6,0) arc (180:0:4);
    \draw          (2,0) arc (0:-180:4);
    \node at (-4.5,2.8) {$x_{\gamma_\infty}$};

    %% \begin{scope}[very thick]
    %% \end{scope}
  \end{tikzpicture}
  \caption{The discriminant points $b_1,\ldots,b_n$ with their
    respective monodromy paths $x_{\gamma_1}, \ldots, x_{\gamma_n}$ and
    the path $x_{\gamma_\infty}$ around the point at infinity.}
  \label{fig: mon}
\end{figure}

Analytically continuing the ordered fibre $(y_0, \ldots, y_{d-1})$
around each of the branch points results in $n+1$ permutations
\[
    \pi_{b_1}, \ldots, \pi_{b_n}, \pi_\infty \in S_d
\]
The group generated by these permutations is called the {\it fundamental
  group} of $\CC \backslash \{b_1, \ldots, b_n\}$. It is denoted
$\pi_1(\PP{1}\CC \backslash \{b_1, \ldots, b_n\}, \alpha_0).$ Observe
that, by the disjoint path condition on the monodromy paths, moving the
base point $\alpha_0$ corresponds to conjugation of the generators of
the fundamental group by some $\pi \in S_d$. Hence, the monodromy group
has explicit dependence on the base point.


%
\subsubsection*{Algorithm}
%

The algorithm implemented in {\tt abelfunctions} for computing the
monodromy group of a curve is based on the one described in
\cite{FKS12}. Due to the technical nature of the algorithm only a
summary is provided here.

\begin{itemize}
  \item We require that the monodromy paths constructed stay
    sufficiently far from the branch points due to the numerical
    accuracy of Newton's method when used in the analytic continuation
    process. For each branch point, $b_i$ we let
    \[
        \rho_i =
        \text{min}_{\substack{j=1,\ldots,n \\ j\neq i}} |b_i - b_j|.
    \]
    The minimal distance that any path $x_\gamma$ be from the branch
    point $b_i$ is
    \[
        R_i = \frac{\rho_i \kappa}{2}
    \]
    where $\kappa \in (0,1]$ is a chosen relaxation factor. The
      implementation of this algorithm in {\tt abelfunctions} uses
      $\kappa = 3/5$.
  \item Let $b = b_i$ be the branch point where $\re b_i < \re b_j$ for
    all $j \neq i$. The point $b$ is referred to as the {\it base branch
      point}. Choose the base point $\alpha_0$ to be the point $\alpha_0
    = b - R_b$, the point on the minimal distance circle encircling $b$.
  \item Order the remaining branch points $\{b_j\}_{j \neq i}$ by
    increasing argument with $b$. This ordering determines the ordering
    of the monodromy paths $x_{\gamma_j}$.
  \item Construct a complete graph $G$ with the branch points
    $b_1,\ldots,b_n$ as nodes and compute the minimal spanning tree $T$
    of this graph with $b$ as the parent node.
  \item Using line segments and semi-circles, construct the path
    $x_{\gamma_j}$ encircling $b_j$ once in the positive direction by
    starting at the base point, following the minimal spanning tree to
    $b_j$ and using semicircles to traverse over or under the branch
    points along the way depending on the ordering. That is, the path
    $x_{\gamma_j}$ should be constructed in such a way so that the
    branch points $b_1,\ldots,b_{j-1}$ lie below the path.
  \item Fix an ordering of the base fibre $(y_0,\ldots,y_{d-1})$ above
    $\alpha_0$ and analytically continue around each $x_{\gamma_j}$ to
    determine the permutations $\pi_j$.
\end{itemize}

%
\subsubsection*{Examples}
%

We compute the monodromy group of the curve
\[
    C : f(x,y) = y^3 + 2x^3y - x^7 = 0,
\]
where the permutations $\pi_j \in \pi_1(\PP{1}\CC \backslash \{b_1,
\ldots, b_n\}, \alpha_0)$ are presented in disjoint cycle notation.
\begin{lstlisting}
from abelfunctions import *
from sympy.abc import x,y,t

f = y**3 + 2*x**3*y - x**7
X = RiemannSurface(f,x,y)

b = X.branch_points()
pi_1 = X.monodromy_group()

for bj,pi_1j in zip(b,pi_1):
    print 'branch point:', bj
    print 'permutation: ', pi_1j
    print
\end{lstlisting}
\begin{pyoutput}
branch point: (-0.31969776999-0.983928563571j)
permutation:  [(0, 2), (1,)]

branch point: (0.836979627962-0.608101294789j)
permutation:  [(0,), (1, 2)]

branch point: (-1.03456371594+0j)
permutation:  [(0,), (1, 2)]

branch point: 0j
permutation:  [(0, 2), (1,)]

branch point: (0.836979627962+0.608101294789j)
permutation:  [(0,), (1, 2)]

branch point: (-0.31969776999+0.983928563571j)
permutation:  [(0, 1), (2,)]

branch point: oo
permutation:  [(0, 2, 1)]
\end{pyoutput}
The method \verb=RiemannSurface.show_paths()= plots all the monodromy
paths $x_{\gamma_j}$ in the complex $x$-plane. The base point
$x=\alpha_0$ is marked in red.
\begin{lstlisting}[firstnumber=14]
X.show_paths()
\end{lstlisting}
\begin{pyoutput}
<matplotlib.figure.Figure object at 0x107d60810>
\end{pyoutput}
\begin{center}
\includegraphics[width=0.75\textwidth]{images/monexample.pdf}
\end{center}


%------------------------------------------------------------------------------
\subsection{Homology} \label{sec: homology}
%------------------------------------------------------------------------------

%
\subsubsection*{Theory}
%

A compact Riemann surface $X$ of genus $g$ is homeomorphic to a sphere
with $g$ handles or, equivalently, a doughnut with $g$ holes. A cycle on
$X$ is a closed, oriented, piecewise smooth curve $\gamma : [0,1] \to X$
such that $\gamma(0) = \gamma(1)$. The first homology group $H_1(X,\ZZ)$
of $X$ is the collection of all cycles on $X$ modulo homologous
transformations. In this document we do not state precisely what it
means for two cycles to be homologous since it involves presenting the
basic theory of simplicial complexes which is beyond the scope of this
document.

However, in brief, two cycles on $X$ are homologous if they can be
deformed to each other where the process of deformation not only allows
continuous transformations but the splitting of one cycle into two via
``pinching''. A demonstration of this procedure is shown in Figure
\ref{fig: pinching}. Two cycles can be added together by ``reversing''
the pinching process and negation of a path corresponds to reversing its
orientation. The {\it first homology group} $H_1(X,\ZZ)$ is the set of
all cycles on $X$ with the addition operation described. The equivalence
of cycles on a Riemann surface is the same as that of closed paths on
the complex plane (specifically, the Riemann sphere) upon which one
integrate a fixed meromorphic function $g$. Closed paths not encircling
a pole of $g$ are homologous to the zero path since they can be
contracted to a point. The set of all paths encircling a single, given
pole are all homologous to each other.

\begin{figure}
  \centering
  %
  % ONE BIG PATH
  \begin{tikzpicture}[scale=0.9]
    \colorlet{darkgreen}{green!50!gray}
    \colorlet{darkblue}{blue!50!gray}

    \begin{scope}[very thick]
    % Quadrant II of Torus
    % (other draw statements are flips / rotations)
    \draw (-6,0) ..
          controls (-6,1.5) and (-5,2.5) ..
          (-3.5,2.5) ..
          controls (-2,2.5) and (-1,1.5) ..
          (0,1.5);
    \draw[xscale=-1] (-6,0) ..
          controls (-6,1.5) and (-5,2.5) ..
          (-3.5,2.5) ..
          controls (-2,2.5) and (-1,1.5) ..
          (0,1.5);
    \draw[rotate=180] (-6,0) ..
          controls (-6,1.5) and (-5,2.5) ..
          (-3.5,2.5) ..
          controls (-2,2.5) and (-1,1.5) ..
          (0,1.5);
    \draw[yscale=-1] (-6,0) ..
          controls (-6,1.5) and (-5,2.5) ..
          (-3.5,2.5) ..
          controls (-2,2.5) and (-1,1.5) ..
          (0,1.5);

    % The Holes
    % (one hole at center shifted to outsides)
    \draw[xshift=-3.2cm] (-0.8,0) ..
          controls (-0.5,0.5) and (0.5,0.5) ..
          (0.8,0);
    \draw[yscale=-1,xshift=-3.2cm] (-1,-0.2) ..
          controls (-0.5,0.5) and (0.5,0.5) ..
          (1,-0.2);

    \draw[xshift=3.2cm] (-0.8,0) ..
          controls (-0.5,0.5) and (0.5,0.5) ..
          (0.8,0);
    \draw[yscale=-1,xshift=3.2cm] (-1,-0.2) ..
          controls (-0.5,0.5) and (0.5,0.5) ..
          (1,-0.2);

    % sum of paths example
    % gamma
    \draw[darkblue, decoration={markings,
              mark=at position 0.25 with {\arrow[very thick]{latex}}},
          postaction={decorate}]
    (0,0) ellipse (5cm and 0.9cm);
    \draw (0,-0.5) node {$\gamma$};
    \end{scope}
  \end{tikzpicture}

  \vspace{24pt}

  %
  % PINCHING THE PATHS
  %
  \begin{tikzpicture}[scale=0.9]
    \colorlet{darkgreen}{green!50!gray}
    \colorlet{darkblue}{blue!50!gray}

    \begin{scope}[very thick]
    % Quadrant II of Torus
    % (other draw statements are flips / rotations)
    \draw (-6,0) ..
          controls (-6,1.5) and (-5,2.5) ..
          (-3.5,2.5) ..
          controls (-2,2.5) and (-1,1.5) ..
          (0,1.5);
    \draw[xscale=-1] (-6,0) ..
          controls (-6,1.5) and (-5,2.5) ..
          (-3.5,2.5) ..
          controls (-2,2.5) and (-1,1.5) ..
          (0,1.5);
    \draw[rotate=180] (-6,0) ..
          controls (-6,1.5) and (-5,2.5) ..
          (-3.5,2.5) ..
          controls (-2,2.5) and (-1,1.5) ..
          (0,1.5);
    \draw[yscale=-1] (-6,0) ..
          controls (-6,1.5) and (-5,2.5) ..
          (-3.5,2.5) ..
          controls (-2,2.5) and (-1,1.5) ..
          (0,1.5);

    % The Holes
    % (one hole at center shifted to outsides)
    \draw[xshift=-3.2cm] (-0.8,0) ..
          controls (-0.5,0.5) and (0.5,0.5) ..
          (0.8,0);
    \draw[yscale=-1,xshift=-3.2cm] (-1,-0.2) ..
          controls (-0.5,0.5) and (0.5,0.5) ..
          (1,-0.2);

    \draw[xshift=3.2cm] (-0.8,0) ..
          controls (-0.5,0.5) and (0.5,0.5) ..
          (0.8,0);
    \draw[yscale=-1,xshift=3.2cm] (-1,-0.2) ..
          controls (-0.5,0.5) and (0.5,0.5) ..
          (1,-0.2);

    % sum of paths example
    % gamma_1
    \draw[darkblue, decoration={markings,
          mark=at position 0.4 with {\arrow[very thick]{latex}}},
          postaction={decorate}]
          (0,0) ..
          controls (3,-2) and (5,-2) ..
          (5,0);
    \draw[darkblue]
          (5,0) ..
          controls (5,2) and (3,2) ..
          (0,0);

    \draw[darkblue, decoration={markings,
          mark=at position 0.4 with {\arrow[very thick]{latex}}},
          postaction={decorate}]
          (0,0) ..
          controls (-3,2) and (-5,2) ..
          (-5,0);
    \draw[darkblue]
          (-5,0) ..
          controls (-5,-2) and (-3,-2) ..
          (0,0);
    \draw (0,-1.1) node {$\gamma$};
    \end{scope}
  \end{tikzpicture}

  \vspace{24pt}

  %
  % SEPARATING THE PATHS
  %
  \begin{tikzpicture}[scale=0.9]
    \colorlet{darkgreen}{green!50!gray}
    \colorlet{darkblue}{blue!50!gray}

    \begin{scope}[very thick]
    % Quadrant II of Torus
    % (other draw statements are flips / rotations)
    \draw (-6,0) ..
          controls (-6,1.5) and (-5,2.5) ..
          (-3.5,2.5) ..
          controls (-2,2.5) and (-1,1.5) ..
          (0,1.5);
    \draw[xscale=-1] (-6,0) ..
          controls (-6,1.5) and (-5,2.5) ..
          (-3.5,2.5) ..
          controls (-2,2.5) and (-1,1.5) ..
          (0,1.5);
    \draw[rotate=180] (-6,0) ..
          controls (-6,1.5) and (-5,2.5) ..
          (-3.5,2.5) ..
          controls (-2,2.5) and (-1,1.5) ..
          (0,1.5);
    \draw[yscale=-1] (-6,0) ..
          controls (-6,1.5) and (-5,2.5) ..
          (-3.5,2.5) ..
          controls (-2,2.5) and (-1,1.5) ..
          (0,1.5);

    % The Holes
    % (one hole at center shifted to outsides)
    \draw[xshift=-3.2cm] (-0.8,0) ..
          controls (-0.5,0.5) and (0.5,0.5) ..
          (0.8,0);
    \draw[yscale=-1,xshift=-3.2cm] (-1,-0.2) ..
          controls (-0.5,0.5) and (0.5,0.5) ..
          (1,-0.2);

    \draw[xshift=3.2cm] (-0.8,0) ..
          controls (-0.5,0.5) and (0.5,0.5) ..
          (0.8,0);
    \draw[yscale=-1,xshift=3.2cm] (-1,-0.2) ..
          controls (-0.5,0.5) and (0.5,0.5) ..
          (1,-0.2);

    % sum of paths example
    \draw[xshift=-3.2cm, darkblue, decoration={markings,
              mark=at position 0.25 with {\arrow[very thick]{latex}}},
          postaction={decorate}]
         (0,0) ellipse (2cm and 1.3cm);
    \draw[xshift=3.2cm, darkblue, decoration={markings,
              mark=at position 0.75 with {\arrow[very thick]{latex}}},
          postaction={decorate}]
         (0,0) ellipse (2cm and 1.3cm);

    \draw (0,-1.1) node {$\gamma = \gamma_1 + \gamma_2$};\
    \draw (-4,1.7)  node {$\gamma_1$};
    \draw (4,1.7)   node {$\gamma_2$};
    \end{scope}
  \end{tikzpicture}
  \caption{A genus $g=2$ Riemann surface $X$ with three homologous
    cycles. The process of ``pinching'' and separating a cycle $\gamma$
    into two cycle is allowed. Cycles can be added together by reversing
    this pinching process. Negation of a cycle corresponds to reversing
    its orientation.}
  \label{fig: pinching}
\end{figure}

$H_1(X,\ZZ)$ has a basis of cycles
$\{a_1,\ldots,a_g,b_1,\ldots,b_g\}$. That is, every cycle on $X$ can be
written as a finite, integer linear combination of the $a$- and
$b$-cycles. These cycles can be chosen such that they satisfy the
intersection properties
\begin{gather*}
  a_i \circ a_j = 0, \quad \forall i \neq j \\
  b_i \circ b_j = 0, \quad \forall i \neq j \\
  a_i \circ b_j = \delta_{ij}, \quad \forall i,j = 1, \ldots, g
\end{gather*}
where $\delta_{ij}$ is the Kronecker delta. That is, the only cycles
that intersect are $a_i$ and $b_i$. A basis of cycles fulfilling these
intersection requirements is called a {\it canonical basis of
  cycles}. Figure \ref{fig: cycle-basis} illustrates the canonical basis
for a genus two Riemann surface.

\begin{figure}
  \centering
  %
  % DEMONSTRATION OF BASIS CYCLES
  %
  \begin{tikzpicture}
    \colorlet{darkgreen}{green!50!gray}
    \colorlet{darkblue}{blue!50!gray}

    \begin{scope}[very thick]
    % Quadrant II of Torus
    % (other draw statements are flips / rotations)
    \draw (-6,0) ..
          controls (-6,1.5) and (-5,2.5) ..
          (-3.5,2.5) ..
          controls (-2,2.5) and (-1,1.5) ..
          (0,1.5);
    \draw[xscale=-1] (-6,0) ..
          controls (-6,1.5) and (-5,2.5) ..
          (-3.5,2.5) ..
          controls (-2,2.5) and (-1,1.5) ..
          (0,1.5);
    \draw[rotate=180] (-6,0) ..
          controls (-6,1.5) and (-5,2.5) ..
          (-3.5,2.5) ..
          controls (-2,2.5) and (-1,1.5) ..
          (0,1.5);
    \draw[yscale=-1] (-6,0) ..
          controls (-6,1.5) and (-5,2.5) ..
          (-3.5,2.5) ..
          controls (-2,2.5) and (-1,1.5) ..
          (0,1.5);

    % The Holes
    % (one hole at center shifted to outsides)
    \draw[xshift=-3.2cm] (-0.8,0) ..
          controls (-0.5,0.5) and (0.5,0.5) ..
          (0.8,0);
    \draw[yscale=-1,xshift=-3.2cm] (-1,-0.2) ..
          controls (-0.5,0.5) and (0.5,0.5) ..
          (1,-0.2);

    \draw[xshift=3.2cm] (-0.8,0) ..
          controls (-0.5,0.5) and (0.5,0.5) ..
          (0.8,0);
    \draw[yscale=-1,xshift=3.2cm] (-1,-0.2) ..
          controls (-0.5,0.5) and (0.5,0.5) ..
          (1,-0.2);

    % a-cycles
    \draw[xshift=-3.2cm, darkblue, decoration={markings,
              mark=at position 0.25 with {\arrow[very thick]{latex}}},
          postaction={decorate}]
         (0,0) ellipse (2cm and 1.3cm);
    \draw[xshift=3.2cm, darkblue, decoration={markings,
              mark=at position 0.25 with {\arrow[very thick]{latex}}},
          postaction={decorate}]
         (0,0) ellipse (2cm and 1.3cm);

    % b-cycles
    \draw[xshift=-3.2cm, yshift=-2.47cm,
          darkgreen, decoration={markings,
              mark=at position 0.25 with {\arrow[very thick]{latex}}},
          postaction={decorate}]
         (0,0) arc (270:90:0.4cm and 1.07cm);
    \draw[xshift=-3.2cm, yshift=-0.33cm, dashed, darkgreen]
         (0,0) arc (90:-90:0.4cm and 1.07cm);
    \draw[xshift=3.2cm, yshift=-2.47cm,
          darkgreen, decoration={markings,
              mark=at position 0.25 with {\arrow[very thick]{latex}}},
          postaction={decorate}]
         (0,0) arc (270:90:0.4cm and 1.07cm);
    \draw[xshift=3.2cm, yshift=-0.33cm, dashed, darkgreen]
         (0,0) arc (90:-90:0.4cm and 1.07cm);

    % cycle labels
    \draw (-4,1.7)  node {$a_1$};
    \draw (4,1.7)   node {$a_2$};
    \draw (-3.2,-3) node {$b_1$};
    \draw (3.2,-3)  node {$b_2$};
    \end{scope}
  \end{tikzpicture}
  \caption{A genus $g=2$ Riemann surface $X$ with the basis cycles
    $\{a_1,a_2,b_1,b_2\}$ for the first homology group $H_1(X,\ZZ)$.}
  \label{fig: cycle-basis}
\end{figure}


%
\subsubsection*{Algorithm}
%

Tretkoff and Tretkoff \cite{TretkoffTretkoff84} provide an algorithm for
determining a canonical cycle basis for $H_1(X,\ZZ)$ given the monodromy
group of a curve. We omit the details of the algorithm here. At its
core, it computes a graph which encodes how to travel from the {\it base
  place} of $X$, chosen to be
\[
    P_0 = (\alpha_0, \beta_0)
\]
where $\alpha_0$ is the base point of the monodromy group of the curve
and $\beta_0 = y_0$ is a fixed root lying above $x = \alpha_0$, to the
other places $(\alpha_0, y_i)$ lying above $x = \alpha_0$ via traversal
around one or more branch points. (These places are sometimes called the
{\it sheets} of the surface $X$ with the $i$th sheet referring to the
place $(\alpha_0,y_i)$). A minimal spanning tree of this graph is
computed where each removed edge corresponds to a cycle in
$H_1(X,\ZZ)$. The is because the addition of an edge to the minimal
spanning tree forms a cycle in the graph. This graph cycle in turn
represents a possibly a non-zero cycle on $X$. A separate part of the
algorithm is then used to compute a canonical basis of cycles by taking
appropriate linear combinations of these intermediate cycles.

%
\subsubsection*{Examples}
%

We compute a homology basis for the Riemann surface $X$ obtained by
desingularizing and compactifying the curve
\[
    C : f(x,y) = y^3 + 2x^3y - x^7 = 0.
\]
The $a$- and $b$- cycles are presented as a list $[\ldots,
  s_i,(b_i,r_i), s_{i+1}, \ldots]$ where $s_i$ is the current sheet
number, $b_i$ is a branch point of $C$, $r_i \in \ZZ$, and $s_{i+1}$ the
the sheet reached after rotating $r_i$ times around $b_i$ and returning
to the base point $\alpha_0$.
\begin{lstlisting}
from abelfunctions import *
from sympy.abc import x,y,t

f = y**3 + 2*x**3*y - x**7
X = RiemannSurface(f,x,y)
a,b = X.homology()

# print the a-cycles
for i in range(g):
    print 'a_%d:'%(i+1)
    print a[i]
    print

# print the b-cycles
for i in range(g):
    print 'b_%d:'%(i+1)
    print b[i]
    print
\end{lstlisting}
\begin{pyoutput}
a_1:
[0, ((-0.31969776999025984-0.9839285635706635j), 1), 2,
 ((-1.0345637159435732+0j), -1), 1,
 ((-0.31969776999025984+0.9839285635706635j), -1), 0]

a_2:
[0, (0j, 1), 2, ((-0.31969776999025984-0.9839285635706635j), -1), 0]

b_1:
[0, ((-0.31969776999025984+0.9839285635706635j), 1), 1,
 ((0.8369796279620464-0.6081012947885316j), 1), 2,
 ((-0.31969776999025984-0.9839285635706635j), -1), 0]


b_2:
[0, ((-0.31969776999025984-0.9839285635706635j), 1), 2,
 ((-1.0345637159435732+0j), -1), 1,
 ((-0.31969776999025984+0.9839285635706635j), -1), 0, (oo, 1), 2,
 ((-0.31969776999025984-0.9839285635706635j), -1), 0,
 ((-0.31969776999025984+0.9839285635706635j), 1), 1,
 ((0.8369796279620464-0.6081012947885316j), 1), 2,
 ((-0.31969776999025984-0.9839285635706635j), -1), 0]
\end{pyoutput}
We can plot the projection of the cycle in the complex $x$- and
$y$-planes. In this example, we plot the cycle $a_1$ by computing 512
interpolating points on the path. The $x$-projection $x_\gamma$ is in
blue and the $y$-projection $y_\gamma$ is in green.
\begin{lstlisting}
alpha = X.base_point()
betas = X.base_lift()
P0 = alpha, betas

gamma = RiemannSurfacePath((f,x,y), P0, cycle = a[0])
gamma.plot(512)
\end{lstlisting}
\begin{pyoutput}
<matplotlib.figure.Figure at 0x106e9cd90>
\end{pyoutput}
\begin{center}
\includegraphics[width=0.75\textwidth]{images/homexample.pdf}
\end{center}

%------------------------------------------------------------------------------
\subsection{Period Matrices} \label{sec: period-matrices}
%------------------------------------------------------------------------------

%
\subsubsection*{Theory}
%

Period matrices are matrices obtained by integrating the holomorphic
differentials $\omega_1, \ldots, \omega_g$ along the $a$-cycles
$a_1,\ldots,a_g$ and $b$-cycles $b_1,\ldots,b_g$. Define the $g \times g$
matrices
\begin{align*}
    A = \left( A_{ij} \right)_{i,j=1}^g,
    \quad A_{ij} = \oint_{a_j} \omega_i, \\
    B = \left( B_{ij} \right)_{i,j=1}^g,
    \quad B_{ij} = \oint_{b_j} \omega_i.
\end{align*}
A {\it period matrix} of $X$ is the $g \times 2g$ matrix
\[
  \tau = \left[ A \; B \right].
\]
We often normalize the differentials $\omega_i$ such that $A_{ij} =
\delta_{ij}$ which results in the period matrix
\begin{equation} \label{eqn: period-matrix}
  \tau = \left[ I_{g \times g} \; \Omega \right].
\end{equation}
This is equivalent to setting $\Omega = A^{-1}B$. The matrix $\Omega \in
\CC^{g \times g}$ is a {\it Riemann matrix}: an invertible, symmetric
complex matrix with positive definite imaginary part. The columns of
$\tau$ define a lattice
\[
    \Lambda = \{I m + \Omega n \; | \; m,n \in \ZZ^g\} \subset \CC^g.
\]
This lattice plays an important role in the theory of algebraic curves
since the quotient space
\begin{equation} \label{eqn: jacobian}
  J(C) = \CC^g / \Lambda \cong \mathbb{T}^{2g}
\end{equation}
is the {\it Jacobian} or {\it Jacobian variety} of the curve
$C$. Jacobian varieties play a central role in the theory of algebraic
curves. For example, the Torelli theorem \cite{Mumford99} states that a
non-singular projective curve is completely determined by its
Jacobian. The Schottky problem establishes a link between the Jacobian
and the Kadomtsev--Petviashvili equation by providing conditions on when
a given Riemann matrix is a period matrix of some algebraic curve.

% Schottky problem

%
\subsubsection*{Algorithm}
%

To compute $A_{ij}, B_{ij}$ we first a method for numerically
integrating holomorphic differentials over any given path $\gamma
\subset C$. We only consider finite paths on the curve: paths with
finite $x$- and $y$-components. Any such path can be parameterized by
some parameter $t \in [0,1]$. Let
\[
    \gamma : [0,1] \to C, \quad \gamma(t) = \Big( x_\gamma(t),
    y_\gamma\big(x_\gamma(t)\big) \Big).
\]
(Recall that we treat $y$ as the dependent variable in $f(x,y)=0$.)
Given this parameterization, we compute the integral of a holomorphic
differential $\omega$. Letting $x$ and $y$ represent the local
coordinates in $\CC^2$ we have
\begin{align}
    \int_\gamma \omega
    &=
    \int_\gamma \omega\big(x,y(x)\big)dx \notag \\
    &=
    \int_0^1 \omega \Big(
    x_\gamma(t), y_\gamma\big(x_\gamma(t)\big) \Big)
    \frac{dx_\gamma}{dt}(t) dt. \label{eqn: path-integral}
\end{align}

In {\tt abelfunctions} we construct paths $\gamma \subset C$ where
$x_\gamma$ either parameterizes a line from some $z_0\in\CC$ to
$z_1\in\CC$
\[
    x_\gamma(t) = z_0(1-t) + z_1t,
\]
or arcs on a circle of radius $R$ with center $w\in\CC$
\[
    x_\gamma(t) = w + Re^{i(\theta + t \Delta \theta)}
\]
where $\theta$ is the starting angle and $\theta + \Delta \theta$ is the
ending angle on the circle. We use the analytic continuation methods
described in Section \ref{sec: analytic-continuation} to compute
$y_\gamma(x_\gamma(t))$. Finally, to compute the integral in Equation
\eqref{eqn: path-integral} we use a numerical integrator of choice. {\tt
  abelfunctions} allows one to use any numerical integrator provided by
the {\tt scipy} Python package that can integrate complex-valued
functions. The Romberg method \cite{wiki:Romberg} is chosen by default.

%
\subsubsection*{Examples}
%

The \verb=RiemannSurface.period_matrix()= method returns the matrices
$A$ and $B$ defined above. The Riemann matrix $\Omega$ is obtained by
computing $\Omega = A^{-1}B$

\begin{lstlisting}
from abelfunctions import *
from sympy.abc import x,y,t
from scipy import dot
from scipy.linalg import inv

f = -x**7 + 2*x**3*y + y**3
X = RiemannSurface(f, x, y)
A,B = X.period_matrix()
Omega = dot(inv(A), B)

print 'A =\n', A
print 'B =\n', B
print 'Omega =\n', Omega
\end{lstlisting}
\begin{pyoutput}
A =
[[ -1.38142275e-12-1.20192474j   1.84957199e+00+0.60096237j]
 [  9.22903420e-12+1.97146395j   7.16176201e-01-0.98573197j]]
B =
[[-0.70647363+2.17430227j -1.84957199+2.54571744j]
 [-1.87497364-1.36224808j -0.71617620+0.23269975j]]
Omega =
[[-1.30901699+0.95105652j -0.80901699+0.58778525j]
 [-0.80901699+0.58778525j -1.00000000+1.1755705j ]]
\end{pyoutput}
We numerically verify that $\Omega$ is a Riemann matrix by computing
$\|\Omega - \Omega^T\|$ as well as the eigenvalues of $\im \Omega$.
\begin{lstlisting}[firstnumber=14]
print norm(Omega.T - Omega)
print
print eigvals(Omega.imag)
\end{lstlisting}
\begin{pyoutput}
9.303308740879998e-11

[ 0.46490467  1.66172235]
\end{pyoutput}

%% %------------------------------------------------------------------------------
%% \subsection{The Abel Map}
%% %------------------------------------------------------------------------------

%% %
%% \subsubsection*{Theory}
%% %

%% The {\it Abel map}, sometimes called the {\it Abel-Jacobi map}, makes
%% concrete the relationship between an algebraic curve $C$ and its
%% Jacobian $J(C)$. Given a place $P \in C$ the Abel map $A : C \to J(C)$
%% is defined
%% \begin{equation} \label{eq: abel-map}
%%   A(P) = \left(
%%   \int_{P_0}^P \omega_1, \ldots, \int_{P_0}^P \omega_g
%%   \right)
%% \end{equation}
%% where $\omega_i, i=1,\ldots g$ are the normalized holomorphic
%% differentials and $P_0$ is some fixed place on $C$.

%% The Abel map is independent of the path chosen from $P_0$ to $P$ since
%% any two paths $\gamma_1$ and $\gamma_2$ form a closed loop on
%% $C$. Therefore, $\gamma_1 - \gamma_2$ is a linear combination of the
%% cycles $a_1,\ldots,a_g,b_1,\ldots,b_g$ and thus the integral of
%% $\omega_i$ is an element of $\Lambda$, which is zero in the quotient
%% space $J(C)$. However, changing the choice of base point $P_0$ changes
%% the map by a translation of the torus.

%% %
%% \subsubsection*{Algorithm}
%% %

%% Given that a method for integrating holomorphic differentials along an
%% arbitrary path $\gamma$ is already established, the primary challenge in
%% computing the Abel map comes from the construction of $\gamma$ itself.


%% %
%% \subsubsection*{Examples}
%% %
